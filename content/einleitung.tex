% !TeX spellcheck = en_US

\chapter{Introduction}
\if 0
индустриальная революция?
Облачные системы всё чаще используются как в больших предприятиях, так и частными лицами (опрос?).
Одной из важнейших причин развития облачных систем считается экономия, гораздо легче и дешевле арендовать часть чужого большого мэйнфрэйма, чем содержать и обслуживать свой маленький сервер. Так же как проще и дешевле отправить небольшую посылку почтой, чем содержать свой автомобиль (сервер) и водителя(администратора) для редких перевозок.
рост популярности облачных систем делает всё более важной тему автоматизации и облегчения управления ими.  Под управлением понимают развертывание, администрирование, обслуживание и финальное свёртывание облачных систем. Одна из часто встречающихся проблем облачных систем является "привязанность", перенос облака настроеного на работу с АПИ одного провайдера, на работы с другим провайдером и другим АПИ является сложной задачей. возможность переноса облаков к более подходящим провайдерам, является залогом развития конкуренции и снежения стоимости содержания облачных систем.
Для решения этой проблемы был разработан стандарт ТОСКА, описывающий  interoperablen und portablen облачные системы и управление ими. Это позволяет облегчить и автоматизировать развёртывание и управление Облачными системами на площадках разных провайдеров. 
ОпенТОСКА это опэн соурс реализация лауфцайтумгэбунг для ТОСКА-приложений разработанное в уни штуттгарт, которое постоянно улучшается и рсширяется.\\
Облачная система ОПэнТОСКИ хранится в формате CSAR, это архив содержащий как описание облачной системы, её внешние функции и связи, так и внутренние пакеты, необходимые для развертывания и работы облачных приложений. Часто это описание содержит ссылки на внешние пакеты и программы, необходимые для развёртывания облачной системы, которые во время разврётывания облачного приложения будут загружены через интернет. Эти загрузки могу внести дополнительные издержки как времени, необходимого на загрузку пакетов, так и денег, потраченных на аренду в холостую работающего сервера, и мегабайты заранее известных пакетов. При наличии одного развёртываемого сервера это может значить несколько секунд промедления, при развертывании большого числа серверов, издержки могут возрасти значительно.
Другой проблемой внешних зависимостей является безопастность и стабильности. Для гарантии безопастности информации, некоторые предприятия ограничивают доступ в интернет для внутренней сети. В других сетях доступ в интернет является крайне ограниченным (например отсутствует широкополосный доступ, медленная связь только по спутнику в определённые часы и тд) Попытка развернуть облачную систему с внешними зависимостями в таких сетях вполне может не увенчатся успехом.  \\
В рамках данной работы необходимо реализовать инструмент для решения проблемы внешних зависимостей. Этот Инструмент должен проанализировать все артефакты  в данном CSAR, идентифицировать внешние зависимости (самый простой пример: найти все команды типа "apt-get install") и устранить их, скачав необходимые пакеты и дополнив ими CSAR. так же он должен адаптировать топологию ТОСКА, для отображения внесённых изменений. Инструмент должен быть легко расширяем, так как невозможно заранее предсказать и описать все возможные типы внешних зависимостей. Результатом работы инструмента является CSAR, содержащий в дополнение к исходной структуре, все необходимые для развертывания облачной системы пакеты и программы, с минимально возможным уровнем обращения к сети интернет во время развёртывания описанной облачной системы. 
  
Структура проработки
%TODO
\fi
In diesem Kapitel steht die Einleitung zu dieser Arbeit.
Sie soll nur als Beispiel dienen und hat nichts mit dem Buch \cite{WSPA} zu tun.
Nun viel Erfolg bei der Arbeit!

Bei \LaTeX\ werden Absätze durch freie Zeilen angegeben.
Da die Arbeit über ein Versionskontrollsystem versioniert wird, ist es sinnvoll, pro \emph{Satz} eine neue Zeile im \texttt{.tex}-Dokument anzufangen.
So kann einfacher ein Vergleich von Versionsständen vorgenommen werden.

\section*{Gliederung}
Die Arbeit ist in folgender Weise gegliedert:
\begin{description}
\item[Kapitel~\ref{chap:basis} -- \nameref{chap:basis}:] Hier werden werden die Grundlagen dieser Arbeit beschrieben.
\item[Kapitel~\ref{chap:zusfas} -- \nameref{chap:zusfas}] fasst die Ergebnisse der Arbeit zusammen und stellt Anknüpfungspunkte vor.
\end{description}
