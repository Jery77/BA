% !TeX spellcheck = en_US

\chapter{Introduction}
\if 0
индустриальная революция?
\fi
Cloud Applications Market increases with great speed. 
Globally annual growth is about 15\%. \cite*{statista_global}
Furthermore observed the growth in the number of firms, which are using Cloud applications. 
And that are not only some big corporations but also many small companies. \cite*{destatis_2014, destatis_2016} \\ 
One of the most important reasons for the development of cloud applications is the economy of resources.
It is much easier and often cheaper to rent a part of another's big mainframe, then to maintain an own server.
As well as it is also easier and cheaper to send a small package by mail, than to keep your own car (server) and driver (administrator) for a rare traffic.\\ 
The growing popularity of cloud applications makes the automation and the ease of management increasingly important.
Under the management is understood the deployment, administration, maintenance and the final roll-off of cloud applications.\\
The common problem of cloud applications is \emph{affection}. 
The transfer of a cloud application configured to interact with the \gls{api} of one provider, to work with another provider and another \gls{api} is a difficult, but important task. 
The ability to quickly move a cloud application to the more suitable provider is a key to the development of competition and reducing the cost of maintenance.\\ %развитие конкуренции
\gls{tosca} \cite*{TOSCA-v1.0} provides an opportunity to solve this problem. 
\gls{tosca} defines the language, which allows describing cloud application and their management portable and interoperable. 
The use of TOSCA allows to simplify and automate the management of cloud applications by different providers. 
According to \gls{tosca} standard a cloud application is stored into \textbf{C}loud \textbf{S}ervice \textbf{AR}chive (CSAR).
This archive contains the description of the cloud application, its external functions and internal dependencies, and the data needed for the deployment and operation.\\
OpenTOSCA \cite*{OpenTOSCA} is an open source ecosystem (runtime environment) for TOSCA standard developed in University of Stuttgart, which is constantly improved and expanded.
OpenTOSCA processes data in CSAR format and performs the actions specified in it.\\
Often these actions contain links to external packages and programs necessary for deployment of the cloud application, which will be subsequently downloaded over the Internet.
This downloads can add expenses to the time required to download packages, money spent on rent an idle server and Internet traffic for megabytes of pre-known data.
If a cloud application consists of only one deployed server, this may mean a few seconds of delay. 
But when an application deploys a large number of linked servers (cloud system), the costs can increase significantly.\\
Other problems of external dependencies are security and stability.
To ensure the security of information, some firms restrict the Internet access.
In other networks, the Internet access is extremely limited.
(For example, there can be no broadband access, slow communication only over a satellite at certain hours, etc)
An attempt to deploy cloud application with external dependencies in such networks may well not succeed. \\
To solve these problems a software solution for resolving external dependencies in CSARs will be developed in implemented during this work.
This software will analyze the CSAR, identify dependencies to external packages and resolve them by downloading the necessary data to install the package (as well as data for all depended packages) and adding them to the CSAR's structure.
The simplest example is to find in given CSAR all the commands like "apt-get install package", delete this command, download the package and all depended packages and add them to the CSAR.\\
This software must be easily expanded (in other words - that will be a framework) since it is impossible to predict and describe all possible types of external dependencies.
The output of the framework is a CSAR, which contains additions to original structure, like all the packages necessary for the deployment of the cloud application, with the minimum possible level of access to the Internet during operation.
\clearpage 
\section*{Structure}
The work is structured as follows:
\begin{description}
\item[Chapter~\ref{chap:basis} -- \nameref{chap:basis}:] This chapter explains the basic terms of this work. These include definitions and descriptions of cloud applications (section \ref{sec:cloud}), TOSCA standard (section \ref{sec:tosca}), OpenTOSCA environment  (section \ref{sec:opentosca}) and Packet management (section \ref{sec:pm}).
\item[Chapter~\ref{chap:req} -- \nameref{chap:req}:] Here are clarified requirements for the framework.
\item[Chapter~\ref{chap:conarch} -- \nameref{chap:conarch}:] In chapter \ref{chap:conarch} the main concepts as well as architecture of the framework are explained and illustrated.
\item[Chapter~\ref{chap:imp} -- \nameref{chap:imp}:] This chapter contains the description of the implementation.
 It explains the design and development of individual components of the framework. 
\item[Chapter~\ref{chap:add} -- \nameref{chap:add}:] New package manager will be added in this chapter, to proof ease of extensibility. 
\item[Chapter~\ref{chap:check} -- \nameref{chap:check}:] Output of the framework will be checked here.
\item[Chapter~\ref{chap:zusfas} -- \nameref{chap:zusfas}] Summarize the results of the work.
\end{description}
