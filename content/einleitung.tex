% !TeX spellcheck = en_US

\chapter{Introduction}
\if 0
индустриальная революция?
\fi
Cloud Applications Market increases with great speed. Globally annual growth is about 15\%. \cite*{statista_global}
Furthermore observed the growth in the number of firms, which use Cloud applications. And that are not only big company but also small firms. \cite*{destatis_2014, destatis_2016} \\ \\
One of the most important reason for development of cloud applications is the saving of resources.
It is much easier and often cheaper to rent a part of someone else's big mainframe, then to maintain your own server.
As well as it is also easier and cheaper to send a small package by mail, than to keep your own car (server) and driver (administrator) for a rare traffic.\\ \\
The growing popularity of cloud applications makes the automation and the ease of management increasingly important.
Under the management is understood the deployment, administration, maintenance and the final coagulation of cloud applications.\\
One common problem of cloud application \emph{affection}. The transfer of a cloud application configured to work with the API of one provider, to work with another provider and another API is a difficult, but important task. The ability to quickly transfer cloud application to more suitable provider, is a key to developing competition and reducing the cost of maintenance.\\ \\
\gls{tosca} \cite*{TOSCA-v1.0} provides a opportunity to solve this problem. \gls{tosca} defines the language, which allows to describe cloud application and they management portable and inter-operable. 
The use of TOSCA allows to simplify and automate the management of cloud applications by different providers. By \gls{tosca} standard a cloud application is stored into \textbf{C}loud \textbf{S}ervice \textbf{AR}chive (CSAR).
This archive contains the description of the cloud application, its external functions and internal dependencies, and the data needed for the deployment and operation of the application\\\\
OpenTOSCA \cite*{OpenTOSCA} is an open source ecosystem (runtime environment) for TOSCA standard developed in University of Stuttgart, which is constantly improved and expanded.
OpenTOSCA processes data in CSAR format and performs the actions specified in it.\\\\
Often these actions contain links to external packages and programs needed to deploy the cloud system, which will be subsequently downloaded over the Internet.
This downloads can add expenses to the time required to download packages, money spent on rent an idle server and megabytes of pre-known data.
If there is one deployed server, this may mean a few seconds of delay, but when deploying a large number of servers, the costs can increase significantly.\\\\
Another problems with external dependencies are security and stability.
To ensure the security of information, some enterprises restrict Internet access for internal networks.
In other networks, Internet access is extremely limited.
(For example there can be no broadband access, slow communication only over a satellite at certain hours, etc)
An attempt to deploy cloud application with external dependencies in such networks may well not succeed. \\\\
As part of this work is necessary to implement a framework for resolving external dependencies in CSARs.
This framework should analyze the CSAR, identify dependencies to external packages and resolve them by downloading this packages (as well as all other dependent packages) and adding them to the CSAR's structure.
The simplest example is to find all the commands like "apt-get install package", delete this command, download the package and all packages necessary to this package, and add them to CSAR.
This framework must be easily expanded, since it is impossible to predict and describe all possible types of external dependencies.
The Result of the framework is a CSAR, which contains additional to original structure, like all the packages necessary for the deployment of the cloud application, with the minimum possible level of access to Internet during operation.

\section*{Structure}
The work is structured as follows:
\begin{description}
\item[Chapter~\ref{chap:basis} -- \nameref{chap:basis}:] This chapter explains the basic terms of this work. These include definitions and descriptions of cloud applications (section \ref{sec:cloud}), TOSCA standard (section \ref{sec:tosca}), CSAR archive (section \ref{sec:csar}), OpenTOSCA environment  (section \ref{sec:opentosca}) and Packet management (section \ref{sec:pm}).
\item[Chapter~\ref{chap:req} -- \nameref{chap:req}:] Here are clarified requirements for the framework.
\item[Chapter~\ref{chap:conarch} -- \nameref{chap:conarch}:] In chapter \ref{chap:conarch} the main concept as well as architecture of the framework are illustrated and explained.
\item[Chapter~\ref{chap:imp} -- \nameref{chap:imp}:] In this chapter contains the description of the implementation. It explains the design and development of individual components of the framework. 
\item[Chapter~\ref{chap:check} -- \nameref{chap:check}:] Output of the framework will be checked here.
\item[Chapter~\ref{chap:zusfas} -- \nameref{chap:zusfas}] Summarize the results of the work.
\end{description}
