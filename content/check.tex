% !TeX spellcheck = en_US

\chapter{Check}\label{chap:check}
\if 0
В этой главе будет протестирован разработанный фреймворк
сначала мы попробуем добавить полученный результаты вайнери и отобразить их
потом установить как то хз как
\fi
\section{Visual check}
\if 0
Для теста коректности полученного после обработки ССАР была установлена вайнери (ссылка на описание). среда для редактирования ТОСКА систем.
надо описать подробнее в базе, со словами - это система хорошо подходит для проверки результатов, опишем её подробнее.
она работает на той платформе линуксовой. Визуальный интерфейс предоставлен в браузере. предоставлен полный набор функций для создания, редактирования и удаления всех возможных элементов системы ТОСКА. может картинок?

обратно к проверке.
Полученный ССАР был добавлен в систему. из за значительного увеличения размера это может являтся довольно продолжительной процедурой. если в изначальном архиве было 10 нод, то после построения дерева зависимостей для пакета питон, их уже стало больше скольки то.
во время добавления проходит проверка на синтаксис ССАР. в случае ошибок будут выведены соответствующие предупреждения.
далее необходимо отобразить сервис темплэйт. Опять таки, из за количества нод предобработка может занять некоторое время, но в то же премя будет выполнена проверка на правильность связей внутри ССАР, если что то было дефинировано неверно, то ячейки или связи между ними не будут отображены.
обработанная тоска система под названием таким то, ранее отображенная на картинке такой то. удаления внешней зависимости по установке питона из ячейки такой то, приобрела следующий вид.
это довольно ненеаглядно, вручную переместив несколько ячеек схема стала несколько проще. проверив несколько ячеек вручную, с помощью команды апт-кэш дэп, можно убедится что связи в тпологии ТОСКИ соотвтетствуют связям в топологии менеджера пакетов.
открыв содержимое ячеек можно убедится что там находятся скрипты для установки пакетов, и сами пакеты для установки.
 
\fi
\section{Installation check}
\if 0
\fi