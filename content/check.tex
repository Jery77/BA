% !TeX spellcheck = en_US

\chapter{Check}\label{chap:check}
In this chapter, the developed framework will be tested.
The resulting CSAR will be added to \nameref{subs:wine} and displayed by the program.
\if 0 
%TODO
надо прописать какой именно архив использовался и какие в нём зависимости
потом установить как то хз как
\fi
\section{Check by Winery}
 \nameref{subs:wine} was installed to test the correctness of output CSAR. 
 This is an environment for development TOSCA systems and will be useful for checking results. \\
 The source CSAR is displayed on figure 
 %TODO
 This CSAR has a purely simple structure.
 THen the output CSAR was added to winery. 
 Due to significant increase in size, this can be a fairly lengthy procedure.
 There where 10 nodes in source CSAR, then after processing bz the framework, there are already more then 100 of nodes.
 During the addition, a CSAR's syntax is tested.
 In case of errors, messages will be displayed.
 Then Service Template will be displayed.
 Again, due to high number of nodes, preprocessing can take a long time. But at the time, the correctness of the links will be checked.
 If something was defined not properly, the nodes or links between them will not be displayed.
 After removing of external references and integrating python's dependency tree, the CSAR has acquired the following form (figure ).
 %TODO
 It's pretty beloved.
 To verify the TOSCA's structure some nodes was moved manually. 
 By checking several nodes with $apt$-$cache$ $depends$ command, the correctness of dependencies can be verified.
 By opening the content of the new nodes, it can be verified, that the are scripts and packages for installation.

\section{Check installation}
Also is is necessary to check whether it is possible to install new packages using the generated installation scripts.
First bash scripts will be tested, then ansible playbooks.
\subsection{Check bash scripts}
\if 0
Проверка сгенерированных баш скриптов это просто. 
Баш является стандартной средой командной строки в линукс, поэтому нужно просто выполнить эти скрипты и проверить корректность установки.
\fi
\subsection{Check ansible playbooks}

\if 0
\fi