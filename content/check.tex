% !TeX spellcheck = en_US

\chapter{Check}\label{chap:check}
In this chapter, the developed framework will be tested.
The resulting CSAR will be added to \nameref{subs:wine} and displayed by the program.
\if 0
потом установить как то хз как
\fi
\section{Check by Winery}
 \nameref{subs:wine} was installed to test the correctness of output CSAR. 
 This is an environment for development TOSCA systems and will be useful for checking results. \\
 Output CSARs where added to winery. 
 Due to significant increase in size, this can be a fairly lengthy procedure.
\if 0

если в изначальном архиве было 10 нод, то после построения дерева зависимостей для пакета питон, их уже стало больше скольки то.
во время добавления проходит проверка на синтаксис ССАР. в случае ошибок будут выведены соответствующие предупреждения.
далее необходимо отобразить сервис темплэйт. Опять таки, из за количества нод предобработка может занять некоторое время, но в то же премя будет выполнена проверка на правильность связей внутри ССАР, если что то было дефинировано неверно, то ячейки или связи между ними не будут отображены.
обработанная тоска система под названием таким то, ранее отображенная на картинке такой то. удаления внешней зависимости по установке питона из ячейки такой то, приобрела следующий вид.
это довольно ненеаглядно, вручную переместив несколько ячеек схема стала несколько проще. проверив несколько ячеек вручную, с помощью команды апт-кэш дэп, можно убедится что связи в тпологии ТОСКИ соотвтетствуют связям в топологии менеджера пакетов.
открыв содержимое ячеек можно убедится что там находятся скрипты для установки пакетов, и сами пакеты для установки.
 
\fi
\section{Check installation scripts}
\if 0
\fi