% !TeX spellcheck = en_US

\chapter{Concept and Architecture}\label{chap:conarch}
\if 0

Концепция
В этом разделе будет разъеснена концепция данной работы

Анализ на внешние ссылки. 
К несчас, тью невозможно выявить все возможные внешние зависимости даже в рамках одного языка, и одного менеджера пакетов (пример какой нить страшный, скрипт нечитаемый вставь), а так как данная работа направленна на создание общего инструмента, легко расширяемого и дополняемого, мы рассмотрим только базовые методы использования менеджеров пакетов, дополняя их по мере необходимости. Эта возможность легко расширять программу, так же будет являться свидетельством корректности её архитектуры в вопросе дополняемости.
В начала логично реализован разбор самой простую и популярную комбинацию, баш скрипт с менеджером пактов apt-get 
Этот простой и мощный инструмент позволяет в одну строчку устанавливать, удалять или обновлять заданный список пакетов.
Когда обрабтка данной комбинации язык-менеджер пакетов будет производиться на приемлимом уровне, можно будет добавлять дополнительные языки и менеджеры пакетов

Архитектура конечного устройства
Самым острым образом стоит вопрос об архитектуре устройства, на котором будут выполняться команды установки. 
К несчастью невозможно проанализировав структуру случайного CSAR, дать однозначный ответ на вопрос, на какой архитектуре какая команда будет выполнена. Здесь кроется множество подводных камней, как то что топология может использовать несколько физических устройств с разными архитектурами, так и то что одни и те же имплемент артифакты(команды установки) могут быть вызваны разными системами. таким образом одна простая команда apt-get install python при запуске на трёх разных устройствах с архитектурами arm, amd64 и i386 приведёт к загрузке и установке 3ёх разных пакетов. Для конечного пользователя это огромное упрощение, но в нашей ситуации это может сильно усложнить анализ.

Были разработаны следующие методы выбора архитектуры. 
(придумать названия)
Скрипт анализирующий систему где он запущен (например с помощью команды унаме -а) и в зависимости от результата устанавливающий пакет соответствующий архитектуре. 

Единая архитектура, заранее заданная пользователем при старте программы

архитектура задаётся для каждого артифакта, устанавливающего что либо

Анализ решений. К несчастью первый вариант, в начале кажущийся наиболее надёжным решением проблемы, несёт с собой множество дополнительных проблем. Так, пакеты для разных архитектур могут отличаться не только сборкой, но и версией, что несёт с собой отличие в зависимостях и как следствие полный хаос при попытке отобразить эти разные пакеты с разными зависимостями и разными версиями в топологии тоски. А попытка создать свободно устанавливаемый архив вполне может не увенчаться успехом. 
Третий вариант, хоть и является более надёжным по сравнению со вторым, несёт в себе дополнительные сложности для пользователя программы, который будет обязан проанализировать каждый артефакт и решить на какой архитектуре он будет запущен, что может не принести результатов, так как уже было указано, что один и тот же артифакт может быть запущен на разных архитектурах.

В результате был выбран второй метод, как самый простой и легкореализуемый. в случае необходимости его легко можно как расширить до третьего (просто заменив выбор единой архитектуры при старте, на выбор её для каждого отдельного артефакта) так и до первого, так как каждый артефакт это уже скрипт установки, который несложным образом можно заменить на более сложный скрипт, определяющий архитектуру, что тем не менее не решит вышеуказанных проблем.


Проблема расширяемости
Фреймоврк должен обрабатывать разные скриптовые языки, каждый из которых может использовать различные пакетные менеджеры.
Для лучшего понимания логично привести малую иллюстрацию иллюстрацию архитектуры, которая будет дополнена и расширена в разделе архитектура
Иллюстрация. программа, много языков, каждый язык включает менеджеры.

Проверка результатов
Проверка результатов работы является важным этапом разработки программы, необходимо удостовериться как в общей валидности выходного CSAR, так и в том что полученный архив будет корректно развёрнут. 
Для проверки на общую правильность возможно использовать инструмент винэри, из комплекта опэнтоски. Этот инструмент для создания и редактирования архивов, также отлично подходит для визуализации результатов.
Проверка развёртки архива это куда более сложная задача, для неё можно воспользоваться каким то там инструментом, но так же нужно создать необходимое окружение для разворачивания облачного приложения
 



Архитектура


\fi