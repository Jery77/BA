% !TeX spellcheck = en_US

\chapter{Basis}
\label{chap:basis}

\section{Cloud Computing}

\if 0
Сейчас Облачные сервера и сервисы можно встретить повсюду, и их число постоянно ратёт (вставь статистику)
Они оспользуются в ... нужны примеры...
История? 
Ранние концепции использования вычислительных ресурсов по принципу системы коммунального хозяйства относят к 1960-м годам (к Джону Маккарти или Джозефу Ликлайдеру)[вставь сам].
Следующими шагами к концептуализации облачных вычислений считаются появление CRM-системы Salesforce.com, предоставляемой по подписке в виде веб-сайта (1999) и начало предоставления услуг по доступу к вычислительным ресурсам через Интернет книжным магазином Amazon.com (2002)[4]. Развитие сервисов Amazon, фактически превратившейся благодаря этим услугам в технологическую компанию, привело к формулировке идеи вычислительной эластичности и запуску в августе 2006 года проекта под названием Elastic Computing Cloud (Amazon EC2). Практически одновременно с запуском ECC термины cloud и cloud computing прозвучали в одном из выступлений главы Google Эрика Шмидта[5][6][7], начиная с этого времени встречаются многочисленные упоминания облачных вычислений в СМИ, в публикациях специалистов по информационным технологиям, в научно-исследовательской среде. Отсылка к «облаку» использовалась как метафора, основанная на изображении Интернета на диаграмме компьютерной сети, или как образ сложной инфраструктуры, за которой скрываются все технические детали.

Запуск в 2009 году приложений Google Apps отмечается как следующий важный шаг к популяризации и осмыслению облачных вычислений[4]. В 2009—2011 годы были сформулированы несколько важных обобщений представлений об облачных вычислениях, в частности, выдвинута модель частных облачных вычислений. В 2011 году Национальный институт стандартов и технологий сформировал определение, которое структурировало и зафиксировало все возникшие к этому времени трактовки и вариации относительно облачных вычислений в едином понятии[8].

определение 
Так как определение выдвиннутое Национального института стандартов и технологий подходящим образом описывает концепцию облачных вычислителей, используемую в данной работе, оно и будет использовано и представленно ниже
\fi
Cloud computing is a model for enabling ubiquitous, convenient, on-demand network access to a shared
pool of configurable computing resources (e.g., networks, servers, storage, applications, and services) that
can be rapidly provisioned and released with minimal management effort or service provider interaction. [nist]
\if 0
НИСТ различает 3 типа моделей сервисов 
\fi
Service Models:
Software as a Service (SaaS). The capability provided to the consumer is to use the provider’s
applications running on a cloud infrastructure2
. The applications are accessible from
various client devices through either a thin client interface, such as a web browser (e.g.,
web-based email), or a program interface. The consumer does not manage or control the
underlying cloud infrastructure including network, servers, operating systems, storage, or
even individual application capabilities, with the possible exception of limited userspecific
application configuration settings.
Platform as a Service (PaaS). The capability provided to the consumer is to deploy onto the cloud
infrastructure consumer-created or acquired applications created using programming languages, libraries, services, and tools supported by the provider.3 The consumer does
not manage or control the underlying cloud infrastructure including network, servers,
operating systems, or storage, but has control over the deployed applications and possibly
configuration settings for the application-hosting environment.
Infrastructure as a Service (IaaS). The capability provided to the consumer is to provision
processing, storage, networks, and other fundamental computing resources where the
consumer is able to deploy and run arbitrary software, which can include operating
systems and applications. The consumer does not manage or control the underlying cloud
infrastructure but has control over operating systems, storage, and deployed applications;
and possibly limited control of select networking components (e.g., host firewalls).

\if 0
и 4 типа моделей размещения 

\fi 
Deployment Models:
Private cloud. The cloud infrastructure is provisioned for exclusive use by a single organization
comprising multiple consumers (e.g., business units). It may be owned, managed, and
operated by the organization, a third party, or some combination of them, and it may exist
on or off premises.
Community cloud. The cloud infrastructure is provisioned for exclusive use by a specific
community of consumers from organizations that have shared concerns (e.g., mission,
security requirements, policy, and compliance considerations). It may be owned,
managed, and operated by one or more of the organizations in the community, a third
party, or some combination of them, and it may exist on or off premises.
Public cloud. The cloud infrastructure is provisioned for open use by the general public. It may be
owned, managed, and operated by a business, academic, or government organization, or
some combination of them. It exists on the premises of the cloud provider.
Hybrid cloud. The cloud infrastructure is a composition of two or more distinct cloud
infrastructures (private, community, or public) that remain unique entities, but are bound
together by standardized or proprietary technology that enables data and application
portability (e.g., cloud bursting for load balancing between clouds). 


\if 0

Облачные вычислители позволяют эффективно использовать ресурсы, распределив нагрузку на систему  из нескольких физических серверов и переложить работу по их обслуживанию на провайдеров.
У пользователя нет прямого доступа к инфраструктуре (серверам и операционным система), он использует только предоставленные провайдером службы и сервисы. 
Набор и функционал этих сервисов и служб зависит от каждого конкретного провайдера и его области специализации, что облегчает работу с этим провайдером. Как недостаток эти Отличия в наборах сервисов и служб затрудняют миграцию от одного провайдера к другому и автоматизацию работы с разными провайдерами.
\fi 

\section{Topology and Orchestration Specification for Cloud	Applications (TOSCA)}
definition
The upcoming OASIS Topology and Orchestration Specification for
Cloud Applications (TOSCA) standard provides new ways to enable portable automated
deployment and management of composite applications. TOSCA describes the
structure of composite applications as topologies containing their components and
their relationships. Plans capture management tasks by orchestrating management
operations exposed by the components.
[TOSCA: Portable Automated Deployment and Management of Cloud Applications]
Usage
\if 0
Тоска может быть использована не только для описания всех стадий жизни- работы облачного приложения, но и служить прослойкой, между облачным приложением и сервисами провайдера, позволяя реализовать единое приложение подходящее для работы с разными провайдерами. 
\fi 
Structure

\section{OpenTOSCA}
\section{CSAR}
\section{Linux Packet management}



