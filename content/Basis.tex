% !TeX spellcheck = en_US

\chapter{Basis}
\label{chap:basis}

\section{Cloud Computing} \label{sec:cloud}

\if 0
Сейчас Облачные сервера и сервисы можно встретить повсюду, и их число постоянно ратёт (вставь статистику)
Они оспользуются в ... нужны примеры...
История? 
Ранние концепции использования вычислительных ресурсов по принципу системы коммунального хозяйства относят к 1960-м годам (к Джону Маккарти или Джозефу Ликлайдеру)[вставь сам].
Следующими шагами к концептуализации облачных вычислений считаются появление CRM-системы Salesforce.com, предоставляемой по подписке в виде веб-сайта (1999) и начало предоставления услуг по доступу к вычислительным ресурсам через Интернет книжным магазином Amazon.com (2002)[4]. Развитие сервисов Amazon, фактически превратившейся благодаря этим услугам в технологическую компанию, привело к формулировке идеи вычислительной эластичности и запуску в августе 2006 года проекта под названием Elastic Computing Cloud (Amazon EC2). Практически одновременно с запуском ECC термины cloud и cloud computing прозвучали в одном из выступлений главы Google Эрика Шмидта[5][6][7], начиная с этого времени встречаются многочисленные упоминания облачных вычислений в СМИ, в публикациях специалистов по информационным технологиям, в научно-исследовательской среде. Отсылка к «облаку» использовалась как метафора, основанная на изображении Интернета на диаграмме компьютерной сети, или как образ сложной инфраструктуры, за которой скрываются все технические детали.

Запуск в 2009 году приложений Google Apps отмечается как следующий важный шаг к популяризации и осмыслению облачных вычислений[4]. В 2009—2011 годы были сформулированы несколько важных обобщений представлений об облачных вычислениях, в частности, выдвинута модель частных облачных вычислений. В 2011 году Национальный институт стандартов и технологий сформировал определение, которое структурировало и зафиксировало все возникшие к этому времени трактовки и вариации относительно облачных вычислений в едином понятии[8].

определение 
Так как определение выдвиннутое Национального института стандартов и технологий подходящим образом описывает концепцию облачных вычислителей, используемую в данной работе, оно и будет использовано и представленно ниже
\fi
Cloud computing is a model for enabling ubiquitous, convenient, on-demand network access to a shared
pool of configurable computing resources (e.g., networks, servers, storage, applications, and services) that
can be rapidly provisioned and released with minimal management effort or service provider interaction. [nist]
\if 0
НИСТ различает 3 типа моделей сервисов 
\fi
Service Models:
Software as a Service (SaaS). The capability provided to the consumer is to use the provider’s
applications running on a cloud infrastructure2
. The applications are accessible from
various client devices through either a thin client interface, such as a web browser (e.g.,
web-based email), or a program interface. The consumer does not manage or control the
underlying cloud infrastructure including network, servers, operating systems, storage, or
even individual application capabilities, with the possible exception of limited userspecific
application configuration settings.
Platform as a Service (PaaS). The capability provided to the consumer is to deploy onto the cloud
infrastructure consumer-created or acquired applications created using programming languages, libraries, services, and tools supported by the provider.3 The consumer does
not manage or control the underlying cloud infrastructure including network, servers,
operating systems, or storage, but has control over the deployed applications and possibly
configuration settings for the application-hosting environment.
Infrastructure as a Service (IaaS). The capability provided to the consumer is to provision
processing, storage, networks, and other fundamental computing resources where the
consumer is able to deploy and run arbitrary software, which can include operating
systems and applications. The consumer does not manage or control the underlying cloud
infrastructure but has control over operating systems, storage, and deployed applications;
and possibly limited control of select networking components (e.g., host firewalls).

\if 0
и 4 типа моделей размещения 

\fi 
Deployment Models:
Private cloud. The cloud infrastructure is provisioned for exclusive use by a single organization
comprising multiple consumers (e.g., business units). It may be owned, managed, and
operated by the organization, a third party, or some combination of them, and it may exist
on or off premises.
Community cloud. The cloud infrastructure is provisioned for exclusive use by a specific
community of consumers from organizations that have shared concerns (e.g., mission,
security requirements, policy, and compliance considerations). It may be owned,
managed, and operated by one or more of the organizations in the community, a third
party, or some combination of them, and it may exist on or off premises.
Public cloud. The cloud infrastructure is provisioned for open use by the general public. It may be
owned, managed, and operated by a business, academic, or government organization, or
some combination of them. It exists on the premises of the cloud provider.
Hybrid cloud. The cloud infrastructure is a composition of two or more distinct cloud
infrastructures (private, community, or public) that remain unique entities, but are bound
together by standardized or proprietary technology that enables data and application
portability (e.g., cloud bursting for load balancing between clouds). 


\if 0

Облачные вычислители позволяют эффективно использовать ресурсы, распределив нагрузку на систему  из нескольких физических серверов и переложить работу по их обслуживанию на провайдеров.
У пользователя нет прямого доступа к инфраструктуре (серверам и операционным система), он использует только предоставленные провайдером службы и сервисы. 
Набор и функционал этих сервисов и служб зависит от каждого конкретного провайдера и его области специализации, что облегчает работу с этим провайдером. Как недостаток эти Отличия в наборах сервисов и служб затрудняют миграцию от одного провайдера к другому и автоматизацию работы с разными провайдерами.
\fi 

\section{Topology and Orchestration Specification for Cloud	Applications (TOSCA)} \label{sec:tosca}
definition
The upcoming OASIS Topology and Orchestration Specification for
Cloud Applications (TOSCA) standard provides new ways to enable portable automated
deployment and management of composite applications. TOSCA describes the
structure of composite applications as topologies containing their components and
their relationships. Plans capture management tasks by orchestrating management
operations exposed by the components.
[TOSCA: Portable Automated Deployment and Management of Cloud Applications]
Usage
\if 0
Тоска может быть использована не только для описания всех стадий жизни- работы облачного приложения, но и служить прослойкой, между облачным приложением и сервисами провайдера, позволяя реализовать единое приложение подходящее для работы с разными провайдерами. 
\fi 
Structure
TOSCA specification provides a language to describe service
5 components and their relationships using a service topology, and it provides for describing the
6 management procedures that create or modify services using orchestration processes. The combination
7 of topology and orchestration in a Service Template describes what is needed to be preserved across
8 deployments in different environments to enable interoperable deployment of cloud services and their
9 management throughout the complete lifecycle (e.g. scaling, patching, monitoring, etc.) when the
10 applications are ported over alternative cloud environments.
[TOSCA-v1.0-cs01.pdf]
 A Topology Template (also referred to as the topology
103 model of a service) defines the structure of a service
A Topology Template consists of a set of Node Templates and Relationship Templates that together
108 define the topology model of a service as a (not necessarily connected) directed graph. A node in this
109 graph is represented by a Node Template. A Node Template specifies the occurrence of a Node Type as
110 a component of a service. A Node Type defines the properties of such a component (via Node Type
111 Properties) and the operations (via Interfaces) available to manipulate the component. Node Types are
112 defined separately for reuse purposes and a Node Template references a Node Type and adds usage
113 constraints, such as how many times the component can occur.
A Relationship Template specifies the occurrence of a relationship between nodes in a Topology
126 Template. Each Relationship Template refers to a Relationship Type that defines the semantics and any
127 properties of the relationship. Relationship Types are defined separately for reuse purposes. The
128 Relationship Template indicates the elements it connects and the direction of the relationship by defining
129 one source and one target element (in nested SourceElement and TargetElement elements). The
130 Relationship Template also defines any constraints with the OPTIONAL
131 RelationshipConstraints element. 
An artifact represents the content needed to realize a deployment such as an executable (e.g. a script, an
217 executable program, an image), a configuration file or data file, or something that might be needed so that
218 another executable can run (e.g. a library). Artifacts can be of different types, for example EJBs or python
219 scripts. The content of an artifact depends on its type. Typically, descriptive metadata will also be
220 provided along with the artifact. This metadata might be needed to properly process the artifact, for
221 example by describing the appropriate execution environment.
222 TOSCA distinguishes two kinds of artifacts: implementation artifacts and deployment artifacts. An
223 implementation artifact represents the executable of an operation of a node type, and a deployment 
artifact represents the executable for materializing instances of a node. For example, a REST operation
225 to store an image can have an implementation artifact that is a WAR file. The node type this REST
226 operation is associated with can have the image itself as a deployment artifact.
227 The fundamental difference between implementation artifacts and deployment artifacts is twofold, namely
228 1. the point in time when the artifact is deployed, and
229 2. by what entity and to where the artifact is deployed.
\section{CSAR} \label{sec:csar}
Ein Cloud Service Archive (CSAR) wird zur Verteilung einer TOSCA-Anwendung
eingesetzt. Es handelt sich um eine ZIP-Datei (üblicherweise komprimiert) mit der
Dateiendung „csar“, die alle Artefakte enthält, die zur Instanziierung und zum
Management der Cloud-Anwendung bzw. Service benötigt werden. Dazu gehören
Definitions-Dokumente, Implementation Artifacts, Deployment Artifacts und Pläne.
In dieser Form kann eine Cloud-Anwendung einer TOSCA-Laufzeitumgebung
übergeben werden. [TOS13]
Das Wurzelverzeichnis einer CSAR muss zumindest die Ordner „Definitions“ und
„TOSCA-Metadata“ enthalten. In „Definitions“ liegen eine oder mehrere Definitions-
Dokumente der Cloud-Anwendung. Beispielsweise wäre es denkbar, dass eine CSAR
zur Wiederverwendung lediglich Node Types und Relationship Types spezifiziert. In
weiteren CSARs könnten dann Node Templates bzw. Relationship Templates definiert
sein, die auf diese Node Types bzw. Relationship Types referenzieren. In diesem
Fall müssten die Definitions-Dokumente mit den Types importiert werden (siehe
Abschnitt 2.2.1). Falls eine Cloud-Anwendung dagegen vollständig in einer CSAR
verteilt wird, so muss mindestens ein Definitions-Dokument ein Service Template
enthalten. [TOS13]
Der Ordner „TOSCA-Metadata“ enthält die TOSCA Metadatei „TOSCA.meta“
(auch CSAR Manifest genannt), in der Metadaten der CSAR als Schlüssel-Wert-Paare
hinterlegt sind. Abbildung 2.1 veranschaulicht den Aufbau der TOSCA Metadatei.
Die Datei teilt sich in Blöcken auf, die durch Leerzeilen voneinander getrennt sind. Im
ersten Block („block_0“) stehen Metadaten über die CSAR selbst, wie z. B. der Autor
der CSAR. Unter dem optionalen Attribut „Entry-Definitions“ kann der relative Pfad
15
2 Grundlagen
zum Haupt-Definitions-Dokument definiert werden, um einem TOSCA-Container eine
effizientere Verarbeitung der Definitions-Dokumente zu ermöglichen. Auch “Topology“
ist optional und referenziert auf ein Bild, dass die Topologie der Cloud-Anwendung
veranschaulicht. Letzteres Attribut ist dabei nicht durch TOSCA spezifiziert. Es wird
lediglich vom TOSCA-Container OpenTOSCA (siehe Abschnitt 2.5) verstanden und
verwendet.
In weiteren Blöcken können Metadaten zu Dateien in der CSAR definiert werden,
z. B. der Hash einer Datei. Falls Metadaten zu einer Datei definiert werden sollen, so
muss neben dem relativen Pfad der Datei mindestens ihr Content-Type2 angegeben
werden. Statt dem relativen Pfad der Datei können auch Pattern definiert werden,
um Metadaten direkt mehreren Dateien zuzuweisen. [TOS13]
Die übrige CSAR-Struktur ist dem Ersteller der CSAR überlassen [TOS13]. Abbildung
2.2 zeigt ein Beispiel einer gültigen CSAR.
\section{OpenTOSCA} \label{sec:opentosca}
OpenTOSCA [AI] ist eine am Institut für Architektur von Anwendungssystemen der Universität
Stuttgart entwickelte Laufzeitumgebung für TOSCA Anwendungen.
Mittels OpenTOSCA können Cloud Service Archives (CSARs) [29] installiert
werden.
Abbildung 1 zeigt die Architektur von OpenTOSCA. Hauptkomponenten des
Containers sind OpenTOSCAControl, IA-Engine und Plan-Engine.
Aufgabe der OpenTOSCAControl Komponente ist es, den Ablauf der Bearbeitung
einer CSAR-Datei zu dirigieren. Sie bietet zudem Funktionen an, die durch
die OpenTOSCA Container API mittels einer REST-Schnittstelle nach außen hin,
unter anderem einer grafischen Benutzerschnittstelle, zur Verfügung gestellt
werden.
Die IA-Engine ist für das Deployment von Implementation Artifacts
zuständig, welche in der CSAR enthaltenen sind. Da Implementation
Artifacts, wie in Kapitel 2.2 beschrieben, verschiedenster Art sein können,
ist die IA-Engine mittels eines Plug-in-Systems realisiert. Dadurch können neue
Plug-ins, welche das eigentliche Deployment der Implementation
Artifacts ausführen, zur Laufzeit hinzugefügt werden. Die Endpunkte der
erfolgreich deployten Implementation Artifacts werden in einer Endpunktdatenbank
gespeichert. Aktuell wird das Deployment von Web Archives
(WAR) auf Tomcat [6] sowie Axis Archives (AAR) auf Apache Axis [1] unterstützt.
The OpenTOSCA ecosystem consists out of three parts, as shown in the figure above:

OpenTOSCA Container, a TOSCA runtime environment
Winery, a graphical modelling TOSCA tool
Vinothek, a self-service portal for the applications available in the container

Winery will be used to show results.
\section{Packet management} \label{sec:pm}

\if 0
Мэнеджер пактов
мэнеждер пакетов это набор софтварных инструментов, которые автоматизируют процесс установки, обновления, конфигурации и удаления компьютерных программ. 
Менеджеры пакетов работают с пакетами, расделёнными программами и данными в архивах. пакеты содержат метаданные, такими как имя программы, назначение, версия, производитель и список зависимостей. 
Менеджеры пакетов служат для управления базой данных пакетов, их зависимостей и версий, для предотвращения ошибочной установки программ и отстутствующих зависимостей. 
Менеджеры пакетов разработаны для устранения необходимости в ручную устанавливать и обновлять пакеты. Это очень полезно для больших систем с тысячами установленных программ и запутанной схемой зависимостей и необходимых версий

пакеты
Скаченный менеджерами пакеты, это архивные файлы, содержащие программу или исходный код для её компиляции и метаданные описывающие что необходимо для её равёртывания и работы 

динамические библиотеки
Computer systems which rely on dynamic library linking, instead of static library linking, share executable libraries of machine instructions across packages and applications. In these systems, complex relationships between different packages requiring different versions of libraries results in a challenge colloquially known as "dependency hell".  Good package management is vital on these systems.

репозитории 
To give users more control over the kinds of software that they are allowing to be installed on their system (and sometimes due to legal or convenience reasons on the distributors' side), software is often downloaded from a number of software repositories.[4]
По умолчанию в юникс системах менеджер пакетов использует официальные репозитории, подходящие для данной операционной системы и архитектуры девайса, но пользователь может так же добавить сторонние репозитории, например репозитории стороннего производителя или репозитории с пакетами для другой архитектуры

Зависимости
Менеджеры пакетов различают 2 вида зависимостей в пакетах req and prereq
зависимость  пакет1 рек пакет2, отображает необходимость наличия необходимого пакета2 для работы пакета1
зависимость пакет1 пререк пакет2, отображает неоьходимость наличия пакета 2 для установки пакета1
зависимоть первого типа, рек, приводит к наличию циклов в структуре зависимостей.
\fi


