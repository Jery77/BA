% !TeX spellcheck = en_US

\chapter{Implementation}\label{chap:imp}
This chapter provides the information about the implementation of the framework and his elements (or modules), which was described in chapter \ref{chap:conarch}.
The Java language was chosen, because of his simplicity and strength. 
In the Java language, the modules and elements are represented by classes.

\section{Global modules}
This section describes the modules used throughout the whole framework's execution.

\subsection*{CSAR handler}
CSAR handler provides an interface to access the CSAR content and stores information about files associated with it.
For example:
\begin{itemize}
	\item The temp extraction folder.
	\item The list of files from the CSAR.
	\item The meta-file entry.
	\item The architecture of The target platform.
\end{itemize}
All this data are encapsulated into CSAR handler.
To access them public the functions can be used.
\begin{itemize}
	\item $unpack$ and $pack$ extract the CSAR to the temp folder and pack the temp folder back to the CSAR. 
	These functions use the $ZIP$~$handler$ module described below.
	\item $getFiles$ returns the list containing all the files presented in the CSAR.
	\item $getFolder$ returns the path to the folder, where the CSAR was extracted.
	\item $getArchitecture$ returns the architecture used for the CSAR.
	\item $addFileToMeta$ adds an information about the new file to the meta-data.
\end{itemize}

\subsection*{Utils}
This class provides methods, used by many other modules.
\begin{itemize}
	\item $createFile(filename, content)$ creates the file with the given content.
	\item $getPathLength(path)$ returns the deep of the path.
	\item $correctName(name)$ adapts the name for use by the OpenTOSCA.
\end{itemize}

\subsection*{Zip handler}
This is a small module with strait functionality. 
It serves to pack and unpack zip archives, which are used by the CSAR standard.
To handle archives it was decided to use the package $java$.$utils$.$zip$.
The functions of archiving and unarchiving are called respectively $zipIt$ and $unZipIt$. The Java declaration of this functions is provided in the listing \ref{lst:zip}
\begin{Listing}
\caption{The common functions to handle zip archives}
\label{lst:zip}
\begin{lstlisting}
/**
* Unzip it
* 
* @param zipFile   input zip file name
* @param outputFolder     output folder
*/ 
static public List<String> unZipIt(String zipFile, String outputFolder);

/**
* Zip all files in folder
* 
* @param zipFile output ZIP file location
* @param folder, containing files to zip
* @throws FileNotFoundException, IOException
*/
static public void zipIt(String zipFile, String folder)
\end{lstlisting}
\end{Listing}

\section{References resolver}
This is the main module which starts by framework startup and is executed into three stages.

\subsection*{Preprocessing}
At the preprocessing stage, the CSAR is unpacked, common \gls{tosca} definitions generated and internal dependencies trees build. 

\subsubsection*{Unpacking}
To unpack the CSAR the function $unpack$ from the CSAR handler is used.

\subsubsection*{Generating TOSCA Definitions}
To generate common \gls{tosca} definitions the $javax$.$xml$.$bind$ package was chosen. 
Descriptions for common definitions were created. (A definition defines the element of TOSCA standard. A description is used to create the definition.)
\begin{itemize}
	\item $DependsOn$ and $PreDependsOn$ describe Relationship Types %(Described in the section \nameref{subs:reltype})
	 for the dependency types between packages.% (described in the section \nameref{subs:dep}). 
	\item $Package$ $Artifact$ describes a deployment Artifact Type for a package installation data.
	\item $Script$ $Artifact$ describes an implementation Artifact Type for scripts installing packages.
	\item $Ansible$ $Playbook$ describes a deployment Artifact Type for a package installation via Ansible playbook.
\end{itemize}
An example description of the $Script$ $Artifact$ can be found in the listing~\ref{lst:scripttype}.
Each description is presented by a separate Java class.

\subsubsection*{Build internal dependencies trees}\label{subs:imp_findintref}
Internal dependencies are mainly used by the \nameref{subs:archtop}.
Therefore, these two modules were combined within the one Java class named $Topology$~$Handler$.	
At the preprocessing stage, it analyses all origin definitions to build internal dependencies trees, as was described in section~\ref{subs:analyse}.
To read origin definitions from the XML files the package $org$.$w3c$.$dom$ was used.

\subsection*{Processing}
During this stage, all described language modules are started.
Since the language modules are stored in $language$ variable, this simple stage can be presented by the listing~\ref{lst:start_lang}.
\begin{Listing}
\caption{The processing stage}
\label{lst:start_lang}
\begin{lstlisting}
for (Language l : languages)
	l.proceed(cr);
\end{lstlisting}
\end{Listing}


\subsection*{Finishing}
To finish the work the changed data should be packed back to \gls{csar}.
The function $pack$ from the CSAR handler is used.

\section{Search for external references} 
This section will describe the search for external references in the original artifacts. %implementation of %TODO
For this purpose serve \nameref{subs:archlm} and \nameref{subs:archpmm}.
Since the framework is initially oriented to easy extensibility, abstract models for \nameref{subs:archlm} and \nameref{subs:archpmm} will be defined.
New languages and package managers can be added by implementing these models.

\subsection*{Language model}
To describe the common functionality and behavior of different language modules, the Language model is used. 
In the Java, this abstract model is described by an abstract class. 
The abstract class $Language$ is presented in the listing~\ref{lst:langabst}.
The common components for all language modules are: 
\begin{itemize}
	\item The name of the language.
	\item The set of package manager modules.
	\item The extensions of files.
\end{itemize}
And the common functions are: 
\begin{itemize}
	\item $getName$ returns the name of this language.
	\item $getExtensions$ returns the list of extensions for this language.
	\item $proceed$ checks all original files and transfers results to package manager modules.
	\item $getNodeName$ returns the name for Node Type, which will install package with this language.
	\item $createTOSCA\_Node$ creates the TOSCA definitions for the package. 
	Since the created package nodes must install a package using the same language as the original node, all languages must provide the method for creating the definitions.
\end{itemize}

\subsection*{Package handler model}
Like to languages, an abstract class for package handlers is defined at first.
His description contains only one function $proceed$ (In the listing~\ref{lst:pmabst}), that finds and eliminates external references, as well as passes the found package names to the package handler.

\subsection*{Bash module implementation}
The processing of popular Bash language was implemented.
As the signs of belonging to the Bash language, the file extension (".sh" and ".bash") and the first line ("\#!/bin/bash") are used. 
All files which satisfy this conditions are passed to package managers modules, in our case - to the $apt$-$get$ module. \\
A Bash package node is defined by Node Type, Node Type Implementation, Package Artifact, Script Artifact.
This package node will be instantiated later by the topology handler.
The definitions are created by $createTOSCA\_Node$ method presented in the listing~\ref{lst:create_bash}.
Consider it in more details.
To avoid creating of the same nodes, the names of created nodes are stored in the $created\_packages$ list.
Then the node name is generated using $getNodeName$ and TOSCA definitions for this name are created.

\subsubsection*{Apt-get Bash implementation}
The apt-get package manager module is a simple line-by-line file parser which searches for the lines starting with the "$apt$-$get$ $install$", comments them out and passes this command's arguments to package handler's public function $getPackage$. 
The code can be found in the listing~\ref{lst:bash_apt_parse}.
\subsection*{Ansible implementation}
To test the extensibility of the framework, the \nameref{lang:ansible} language was added.
Since ansible playbooks are often packed to archives, therefore it may be necessary to unpack them first and then analyze the content.
Thus, the files are either immediately transferred to the package handler, or they are unzipped first.
Listing~\ref{lst:ansible_proceed} presents these operations.
As a sign of the ansible language, the ".$yml$" extension is used, since its playbooks don't contain any specific header.\\
Creating a \gls{tosca} node for this language is a complicated operation. 
The basic moments are:
\begin{itemize}
	\item Analyze original files to determine the ansible configuration (the set of options like username or proxy).
	\item It can be necessary to complement the configuration using a user input.
	\item Create the folder with necessary files (the executable $.yml$ file and a subfolder with the installation data).
	\item Pack these files to the Zip file.
	\item Create TOSCA definitions of the package node. The ansible package node is defined by Node Type, Node Type Implementation, and Ansible Artifact.
\end{itemize} 

\subsubsection*{Apt implementation}
Since the package installation written in the $ansible$ language with the $apt$ package manager can be described in many different ways, then the processing will be a complicated task too.
It's worth mentioning that the processing uses a simple state machine and regular expression from the $java$.$util$.$regex$ package.

\section{Package Handler}
Package handler provides an interface for interaction with the package manager of the operating system.
It allows to load packages and to determine the type of dependencies between them.

\subsection*{Package downloading}
This operation is performed using one recursive function $getPacket$ defined in the listing \ref{lst:getpack}.
\begin{Listing}
\caption{The $getPackage$ definition}
\label{lst:getpack}
\begin{lstlisting}
/**
* Download package and check its dependency
* 
* @param language,  language name
* @param packet, package name
* @param listed, list with already included packages
* @param source, name of package or file depending on the package
* @param sourcefile, name of original file contained external reference.
* @throws JAXBException
* @throws IOException
*/
public void getPacket(Language language, String packet, List<String> listed, String source, String sourcefile)
\end{lstlisting}
\end{Listing}
This function downloads packages for the dependency three, calls the language's function $createTOSCA\_Node$ to create package nodes and the topology handler's functions $addDependencyToPacket$ and $addDependencyToArtifact$ to update the topology.\\
The Arguments of the $getPacket$ function will be described shortly.
\begin{itemize}
	\item $language$ is used to call the right $createTOSCA\_Node$ function.
	\item $packet$ is a package name to be downloaded.
	\item $listed$ holds a list with packages already presented in the dependency tree. No need to download them again, but new dependencies will be created.
	\item $source$ defines the parent element in the dependency tree. For the root package that will be the original artifact file, for other packages - the depending package.
	\item $sourcefile$ is the name of the original artifact with external dependencies. This name will be used by the $language$ to generate package node and by topology handler to create the dependency. 
\end{itemize}
For downloading the command $apt$-$get$ $download$ \emph{package} is used. 
If a download fails then the user input is used to solve the problem. 

\subsection*{Dependencies}
To determine the dependency type the command $apt$-$cache$ $depends$  \emph{package} is used.
Example output was presented in section \ref{subs:dep}.

\section{Topology handling}
Topology handler serves to update the TOSCA topology.
For this purpose the \nameref{subs:imp_findintref} is executed during preprocessing stage.

\subsection*{Update Service Templates}
To update Service Templates two functions are provided.
\begin{itemize}
	\item $addDependencyToPacket(sourcePacket,targetPacket,dependencyType)$ generates dependency between two package nodes.
	\item $addDependencyToArtifact(sourceArtifact, targetPacket)$ generates dependency between original node and package node.
\end{itemize} 
The both functions finds all Node Templates and Service Templates for the given $sourcePacket$ or $sourceArtifact$ using the internal dependencies trees.
For each found Node Templates a package node for the $targetPacket$ package is instantiated by creating new Node Template.
Then the dependencies between found Node Templates and new Node Templates is created by instantiating Relationship Templates.
The type of dependency is the value of the $dependencyType$ for $addDependencyToPacket$ and always the $preDependsOn$ for $addDependencyToArtifact$.\\
To update existing TOSCA definition the $org$.$w3c$.$dom$ and $org$.$xml$.$sax$ packages are used. 
Creating of new Node Template is presented in the listing \ref{lst:newnodetemp}.
\begin{Listing}
	\caption{Creating of new Node Template}
	\label{lst:newnodetemp}
	\begin{lstlisting}  
	Element template = document.createElement("tosca_ns:NodeTemplate");
	template.setAttribute("xmlns:RRnt", RR_NodeType.Definitions.NodeType.targetNamespace);
	template.setAttribute("id", getID(package));
	template.setAttribute("name", package);
	template.setAttribute("type", "RRnt:" + RR_NodeType.getTypeName(package));
	topology.appendChild(template);
	\end{lstlisting}
\end{Listing}
