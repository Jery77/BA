% !TeX spellcheck = en_US

\chapter{Requirements}
\label{chap:req}
Since the main purpose of the developed framework is to Resolve References, further the $RR$ can be used as an abbreviation. 
$RR$ should eliminate external dependencies in a TOSCA topology represented by a CSAR file.
$RR$ must be easily extendable to provide the ability to eliminate a large number of dependency types.\\
As a first step, a minimal configuration which handles $Bash$ language with the $apt$-$get$ package manager and $Ansible$ language with the $apt$ package manager will be developed. 
These software handlers of languages and package managers will be called language modules and package manager modules.
As an example, the $Bash$ and $apt$-$get$ modules will remove package installation commands from bash-scripts ($apt$-$get$ $install$ \textbf{$package$}).
Then both the \textbf{$package$} itself and all the depended packages from his dependencies tree will be downloaded.
It is also necessary to update the topology of the TOSCA, by adding new nodes and dependencies.
To do so, common definitions will be added, like Relationship Types and Artifact Types.
 % for downloaded packages and dependencies from nodes previously containing an external reference to the node displaying the downloaded packages. 
Then new nodes will be defined by Node Types, Node Type Implementations, Artifacts Templates, and instantiated by Node Templates. 
Relations between nodes will be instantiated by Relationship Templates.
These Templates must be added to the right Service templates, where the nodes containing external references are instantiated.
%Dependencies between downloaded packages, representing dependency tree should be added too.
To find the Service Templates and Node Types corresponding to a certain artifact, it can be useful to apply preprocessing to the entire TOSCA topology. \\
%References chain can be build:\\
%$script$ $\rightarrow$ $Artifact$ $Implementation$ $\rightarrow$ $Node$ $Type$ $Implementation$ $\rightarrow$ $Node$ $Type$ $\rightarrow$ $Node$ $Template$ $\rightarrow$ $Service$ $Template$\\
After implementing the minimal configuration, it should be easy to add more language modules and package manager modules, like $Aptitude$ for Bash or completely new language like $Chef$.
In order to proof the correctness of the corresponding TOSCA topology, Winery described in section \ref{tool:winery} will be used.
\section*{Stages of the processing}
Here an example is provided, representing how the framework should work.
\begin{itemize}  
	\item Begin  \\
%	To start the work $RR$ needs input CSAR name, output CSAR name, and architecture of target hardware. 
%	This will be done using user input.
	An input CSAR will be extracted.
	\item Preprocessing\\
	During preprocessing stage, RR needs to analyze internal references.
	In additional, common Tosca definitions for artifacts and relations between packages will be added.
	\item Processing with language modules\\
	Each file from the input CSAR will be processed by Language modules.
	\item Processing with packet manager modules\\
    If the file belongs to an Language, it will be processed by the packet manager module belonging to the Language to find and resolve external references.
    Package name from this reference will be moved forward.
	\item Package handling\\
	Using the package name the package will be downloaded and TOSCA definitions created. These actions will be recursively repeated for all dependent packages, creating the dependency tree in the TOSCA topology.
	\item Topology handling\\
	Using information about internal references and dependencies the TOSCA Topology will be updated by creating new Node and Reference Templates. 
	\item End\\
	Meta-file should be updated and all data packed back to the CSAR.
\end{itemize}
These steps will be represented by the modules described in section \ref{sec:arch} and implemented in chapter \ref{chap:imp}.

\section*{Result}
As a result of the work, an output CSAR will be received. 
This CSAR must have the same functionality as the input CSAR, but all external references to additional packages must be resolved.
The output CSAR must be able to be deployed properly without downloading these packages over the Internet. 
In additional, the topology for the packages must be mirrored from the package manager's database to the TOSCA topology.