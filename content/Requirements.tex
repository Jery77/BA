% !TeX spellcheck = en_US

\chapter{Requirements}
\label{chap:req}
Since the main purpose of the developed program is to Resolve References, further the RR will be used as an abbreviation. 
A framework should be developed that eliminates external dependencies in TOSCA topology represented in CSAR file.
The framework should be easy extendable to provide the ability to eliminate a large number of dependency types.
At first a minimal configuration will be developed, that finds all the bash scripts which use the apt-get package manager. 
All package installation commands should be removed (apt-get install package).
Then both the package itself and all the packages from his dependency tree should be downloaded.
It is also necessary to update the topology of the TOSCA, by adding nodes for downloaded packages and dependencies from nodes previously containing an external reference to the cells displaying the downloaded packages. Dependencies between downloaded packages, representing dependency tree should be added too.
In the TOSCA language: new definitions for given package should be created, like Node Type, NodeType Implementation and Artifacts. 
This definitions will be instantiated by adding Node Template and Relationship Template to the right Service Template.
To find a Service Templates and Node Types corresponding to a certain artifact, it will be useful to use preprocessing of entire TOSCA topology. References chain can be build:\\
$script$ $\rightarrow$ $Artifact$ $Implementation$ $\rightarrow$ $Node$ $Type$ $Implementation$ $\rightarrow$ $Node$ $Type$ $\rightarrow$ $Node$ $Template$ $\rightarrow$ $Service$ $Template$\\
In additional to the bash, it should be easy to add additional script languages and package managers, like $aptitude$ for bash or new languages like $shef$ or $ansible$.
To proof the correctness of the corresponding TOSCA topology a winery described in section \ref{tool:winery} can be used.
\section*{Stages of CSAR processing}
Here an example steps are provided, representing how the framework should work.
\begin{itemize}  
	\item Begin  \\
	To start the work our program need to became input CSAR name, output CSAR name, and architecture of target hardware. 
	Then input CSAR need to be extracted.
	\item Preprocessing\\
	During preprocessing stage, RR need to analyze internal references and build dependency three.
	Furthermore, common Tosca definitions for artifacts and  relations between packages need to be added.
	\item Processing with Languages
	Each file from CSAR need to be processed by each Language handler, described in RR.
	\item Processing with Packet Managers
    If file belongs to the Language, he will be processed by Packet Manager Handler to find and resolve external references.
    Package name from this reference will be provided to package handler.
	\item Package handling
	After Package Handler receives package name he must download the right package, create TOSCA definition, send request to Topology Handler  and recursively repeats this actions for all dependent packages, creating Dependency Tree.
	\item Topology handling
	Topology Handler use information about internal references and dependencies to update TOSCA Topology by creating Node and Reference Templates. 
	\item End
	To complete processing Meta-file will be updated and Data packed back to the CSAR.
	
\end{itemize}