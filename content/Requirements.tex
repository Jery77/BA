% !TeX spellcheck = en_US

\chapter{Requirements}\label{chap:req}
In this chapter, requirements for the program and its components being developed will be defined.
Furthermore it will be described what we expect as a result of the application's work.\\
It's necessary to develop new software, which will resolve external references and complement the TOSCA topology to represent the changes in a given CSAR file.
Since the main purpose of the software is to resolve references, further it can be called $References$ $resolver$ $framework$.  
During this work, such terms as input CSAR and output CSAR will be used.
The input \gls{csar} is the \gls{csar}, which can contain external references and will be processed by the framework. %\\
The output CSAR is the CSAR, which was processed by the framework and doesn't contain external references. \\%(at least those that are handled by the framework).\\
The application must identify references to external packages.
That includes a wide range of package installation command types which contact package repositories with the help of different package managers.
To find such commands the software must be able to unpack the data and access all artifacts contained in a input CSAR .
Since one can't develop an application which support all possible types of external references the structure of the application must ensure an easy extensibility which allows to extend the software and handle any new types of references.
Therefore the framework must be developed in the form of modular system so that each module will be able to identify and resolve the corresponding external reference type.
The two types of modules must be developed.
One type of modules will be responsible for configuration management tool and the other type will process package managers.\\
When an external reference is identified it must be resolved. T
hat includes remove of package installation command and addition of the package into the proceeded CSAR.
To provide the encapsulation of the output CSAR the package must be accompanied by all dependent packages which must be found, downloaded and added to the CSAR too.
Then all downloaded packages must be integrated into the TOSCA topology.
That means that one must define new nodes and relationships to map the dependencies from a package database into the topology.
As a last step the extracted input CSAR with additions must be packed to the output CSAR.
The concept and architecture of a program which satisfies the requirements will be described in chapter~\ref{chap:conarch}.
The implementation of the program is provided in chapter~\ref{chap:imp}.
\if 0 
A primary type of external references is an artifact represented by a bash script which uses an apt-get package manager to install new packages.
The exact command for this type of references is "$apt$-$get$~$install$ \textbf{package}".
In this example, References resolver should comment out such commands and add the installation of the \textbf{package} and installations of all packages from its dependencies tree as new separate nodes to the TOSCA topology. \\
A system package manager can be used for this purpose. 
An example for the $apt$-$get$ package manager is provided. 
Using the "$apt$-$get$~$download$ \textbf{package}" command, the installation data for the \textbf{package} can be downloaded, and using the "$apt$-$cache$ $depends$ \textbf{package}" command, the list of dependencies for the \textbf{package} can be obtained. 
To install a package from the installation \textbf{data} the simple command "$dpkg$~$-i$~\textbf{data}" can be used.\\
An installation data should be integrated into the TOSCA topology. 
A new node should be created for each package. 
During this, step a set of definitions will be created.
A common description of a new node is provided via a Node Type, which contains the "install" operation. 
This operation will be implemented by a Node Type Implementation, which uses Artifact Templates.
The created nodes should install packages using the same language, as the original node contained the external reference.
In the example with the Bash language, the Artifact Templates will be represented by the deployment artifact with the installation data and by the implementation artifact with the script containing the installation command. \\
Then the nodes should be instantiated and referenced.
For each Node Template which uses an artifact with external references to a \textbf{package}, a separate Node Template of the \textbf{package}'s Node Type should be defined and referenced.
Then for each \textbf{package}'s Node Template additional Node Templates for packages from its dependencies tree should be defined and referenced.
To define a reference between nodes a Relationship Template will be used and to find out which Node Template uses which artifact the CSAR will be preprocessed.
During the preprocessing the topology of the SCAR will be analyzed and internal references found.\\
To distinguish between languages, package managers and they software handlers, a program component handling a language will be called the language module, and the one handling a package manager - the package manager module.
The Bash module with the apt-get module will be implemented first. 
The goal is to develop extendable software where new modules can easily be added later.
This type of a software is called a framework.\\
%$RR$ must be  to provide the ability to eliminate a large number of dependency types.\\
After the minimal configuration with the $Bash$ and $apt$-$get$ modules is developed, an $Ansible$ module with $apt$ package manager module can be added. 
Ansible scripts are called playbooks. 
Ansible playbooks and related data are often packed to a zip archive for encapsulation.
That makes the ansible module harder to implement since it should not only parse playbooks but additionally unpack archives. Ansible Node Type Implementations will contain only one artifact.
This artifact will be an archive containing both the playbook and the installation data.
%\fi
\section*{Handling}
There is an example, representing how the framework should process a CSAR.\\
At start an input CSAR will be extracted to a temporary folder to handle its content.
Then the internal structure will be analyzed during the preprocessing. 
In addition, the common TOSCA definitions not belonging to a specific node (like Artifact Types or Relationship Types) will be added. 
After that the processing starts. 
Each file from the input CSAR will be processed by each language module.
If a language module accepts a file, then the file is transferred to the package manager modules belonging to the language module.
A package manager module will resolve reference by commenting out the installation command and extracting package names from the command.
Using the package name the package installation data will be downloaded, the installation script and the TOSCA node created. 
To create a new TOSCA node for the given package, the definitions for Node Type, Node Type Implementation, and Artifact Templates should be added to the CSAR. 
A separate Node Template, as well as Relationship Template, will be defined for each depending node.
These actions will be recursively repeated for all depending packages from the package's dependencies tree which mirrors the tree to the TOSCA topology.
At the end, the meta-file should be updated and the data packed back to the SCAR. %\\
This behavior will be described in the chapter \ref{chap:conarch} and implemented in the chapter \ref{chap:imp} in more detail.
\fi
\section*{Result}
As a result of the program's work, an output CSAR will be received.      
This CSAR must have the same functionality as the input CSAR, but all external references to additional packages must be resolved.
The output CSAR must be able to be deployed properly without downloading these packages over the Internet. 
In addition, the dependencies trees for packages from new nodes should be represented in the TOSCA topology. \\
In order to validate the output CSAR, the TOSCA topology can be verified and the defined artifacts can be validated through those installations on a test machine. 
%\clearpage
\section*{Summarize}
The requirements are summarized here. 
The developed framework must:
\begin{enumerate}
	\item identify references to external packages in a input CSAR.\label{req:identify}
	\item delete the external references.\label{req:delete}
	\item add data into the CSAR to encapsulated installations of the packages.\label{req:adddata}
	\item handle different configuration management tools and package managers.\label{req:handledif}
	\item have an easy expandable structure.\label{req:expand}
	\item represent the packages into the TOSCA topology.\label{req:represent}
	\item generate an output CSAR with the same functionality as the input CSAR, but without external references.\label{req:out}
\end{enumerate}
