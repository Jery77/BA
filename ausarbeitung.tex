% !TeX document-id = {7f59beb7-c93a-43bd-a036-7847a50120d2}
% !TeX spellcheck = en_US
% !TeX program = make
% Dieses Dokument muss mit PDFLatex gesetzt werden
% Vorteil: Grafiken koennen als jpg, png, ... verwendet werden
%          und die Links im Dokument sind auch gleich richtig
%
%Ermöglicht \\ bei der Titelseite (z.B. bei supervisor)
%Siehe https://github.com/latextemplates/uni-stuttgart-cs-cover/issues/4
\RequirePackage{kvoptions-patch}

%English:
\let\ifdeutsch\iffalse
\let\ifenglisch\iftrue

%German:
%\let\ifdeutsch\iftrue
%\let\ifenglisch\iffalse

%
\ifenglisch
	\PassOptionsToClass{numbers=noenddot}{scrbook}
\else
	%()Aus scrguide.pdf - der Dokumentation von KOMA-Script)
	%Nach DUDEN steht in Gliederungen, in denen ausschließlich arabische Ziffern für die Nummerierung
	%verwendet werden, am Ende der Gliederungsnummern kein abschließender Punkt
	%(siehe [DUD96, R3]). Wird hingegen innerhalb der Gliederung auch mit römischen Zahlen
	%oder Groß- oder Kleinbuchstaben gearbeitet, so steht am Ende aller Gliederungsnummern ein
	%abschließender Punkt (siehe [DUD96, R4])
	\PassOptionsToClass{numbers=autoendperiod}{scrbook}
\fi

%Warns about outdated packages and missing caption delcarations
%See https://www.ctan.org/pkg/nag
\RequirePackage[l2tabu, orthodox]{nag}

%Neue deutsche Trennmuster
%Siehe http://www.ctan.org/pkg/dehyph-exptl und http://projekte.dante.de/Trennmuster/WebHome
%Nur für pdflatex, nicht für lualatex
\RequirePackage{ifluatex}
\ifluatex
%do not load anything
\else
	\ifdeutsch
		\RequirePackage[ngerman=ngerman-x-latest]{hyphsubst}
	\fi
\fi

\documentclass[
               fontsize=12pt, %Default: 11pt, bei Linux Libertine zu klein zum Lesen
% BEGINN: Optionen für typearea
               paper=a4,
               twoside,  % fuer die Betrachtung am Schirm ungeschickt
               BCOR=3mm, % Bindekorrektur
               DIV=13,   % je höher der DIV-Wert, desto mehr geht auf eine Seite. Gute werde sind zwischen DIV=12 und DIV=15
               headinclude=true,
               footinclude=false,
% ENDE: Optionen für typearea
%               titlepage,
               bibliography=totoc,
%               idxtotoc,   %Index ins Inhaltsverzeichnis
%                liststotoc, %List of X ins Inhaltsverzeichnis, mit liststotocnumbered werden die Abbildungsverzeichnisse nummeriert
               headsepline,
               cleardoublepage=empty,
               parskip=half,
%               draft    % um zu sehen, wo noch nachgebessert werden muss - wichtig, da Bindungskorrektur mit drin
               final   % ACHTUNG! - in pagestyle.tex noch Seitenstil anpassen
               ]{scrbook}


\input{preambel/packages_and_options}

%Der untere Rand darf "flattern"
\raggedbottom

%%%
% Wie tief wird das Inhaltsverzeichnis aufgeschlüsselt
% 0 --\chapter
% 1 --\section % fuer kuerzeres Inhaltsverzeichnis verwenden - oder minitoc benutzen
% 2 --\subsection
% 3 --\subsubsection
% 4 --\paragraph
\setcounter{tocdepth}{1}
%
%%%

\makeindex

%Angaben in die PDF-Infos uebernehmen
\makeatletter
\hypersetup{
            pdftitle={}, %Titel der Arbeit
            pdfauthor={}, %Author
            pdfkeywords={}, % CR-Klassifikation und ggf. weitere Stichworte
            pdfsubject={}
}
\makeatother

% Hier stehen alle Abkürzungen
\newacronym{er}{ER}{error rate}
\newacronym{tosca}{TOSCA}{\textbf{T}opology and \textbf{O}rchestration \textbf{S}pecification for \textbf{C}loud \textbf{A}pplication }
\newacronym{fr}{FR}{Fehlerrate}
\newacronym[plural={RDBMS},shortplural={RDBMS}]{rdbms}{RDBMS}{Relational Database Management System}
\newacronym{csar}{CSAR}{\textbf{C}loud \textbf{S}ervice \textbf{AR}chive }
\newacronym{ch}{CH}{\textbf{C}SAR \textbf{H}andler }


\usepackage{titlesec}
\titlespacing*{\subsubsection}{7pt}{0ex}{0ex}
\begin{document}

%tex4ht-Konvertierung verschönern
\iftex4ht
% tell tex4ht to create picures also for formulas starting with '$'
% WARNING: a tex4ht run now takes forever!
\Configure{$}{\PicMath}{\EndPicMath}{} 
%$ % <- syntax highlighting fix for emacs
\Css{body {text-align:justify;}}

%conversion of .pdf to .png
\Configure{graphics*}  
         {pdf}  
         {\Needs{"convert \csname Gin@base\endcsname.pdf  
                               \csname Gin@base\endcsname.png"}%  
          \Picture[pict]{\csname Gin@base\endcsname.png}%  
         }  
\fi

%Tipp von http://goemonx.blogspot.de/2012/01/pdflatex-ligaturen-und-copynpaste.html
%siehe auch http://tex.stackexchange.com/questions/4397/make-ligatures-in-linux-libertine-copyable-and-searchable
%
%ONLY WORKS ON MiKTeX
%On other systems, download glyphtounicode.tex from http://pdftex.sarovar.org/misc/
%
\input glyphtounicode.tex
\pdfgentounicode=1

%\VerbatimFootnotes %verbatim text in Fußnoten erlauben. Geht normalerweise nicht.

\input{macros/commands}
\pagenumbering{arabic}
\Titelblatt

%Eigener Seitenstil fuer die Kurzfassung und das Inhaltsverzeichnis
\deftripstyle{preamble}{}{}{}{}{}{\pagemark}
%Doku zu deftripstyle: scrguide.pdf
\pagestyle{preamble}
\renewcommand*{\chapterpagestyle}{preamble}

\renewcommand{\chapterheadstartvskip}{\vspace{0em}}

%Kurzfassung / abstract
%auch im Stil vom Inhaltsverzeichnis
\ifdeutsch
\section*{Kurzfassung}
\else
\section*{Abstract}
\fi
In recent years, Cloud Computing is gaining more and more popularity. 
%An application designed according to the principle of Cloud Computing are executed on an extern platform and is called a Cloud Application. 
If someone will try to create a Cloud Application suitable to work with several different platforms, he will face a problem.
The problem is that each platform provides its own \textbf{A}pplication \textbf{P}rogramming \textbf{I}nterface (API) to interact with Cloud Applications and it's difficult to create one unified application functioning on various platforms properly. 
\textbf{T}opology and \textbf{O}rchestration \textbf{S}pecification for \textbf{C}loud \textbf{A}pplication (TOSCA) provides a solution for this problem.
%%This standard adds an additional level of abstraction to a Cloud Applications, in other words, a layer between the external interfaces of the Cloud Application and a Cloud service provider's API.
With the help of TOSCA, it's possible to describe several models of interaction with many different APIs within one TOSCA Cloud Application.
A TOSCA runtime environment is responsible to choose and process the right model and serves as a layer between external interfaces of a Cloud Application and an API of a platform.
This allows to automate the migration of Cloud Applications between platforms which use completely different APIs. 
The description of a Cloud Application is stored in a \textbf{C}loud \textbf{S}ervice \textbf{AR}chive (CSAR), which contains all components necessary for the Cloud Application life-cycle. \\
%The University of Stuttgart implemented this specification in the runtime environment named OpenTOSCA. \\
Cloud Applications are often described in such way that during their deployment some external packages, programs and files need to be downloaded via the Internet.
These downloads can slow down the deployment and when the access to the Internet is limited, unstable or missing, they can prevent the installation at all.
%If a Cloud Application consists of a single virtual server with one operating system, this can slightly slow down the deployment. 
%But in composite Cloud Computing, a large number of identical operating systems can download a huge amount of the same data, which can significantly increase the time needed for deployment.\\
%During this work a software solution which will eliminate external dependencies in CSAR, resupply them with all packages necessary for deployment and also change the internal structure to display the achieved self-containment will be developed and implemented.
%For example, all commonly used "apt-get install" commands, which download and install packages, must be removed. 
%Appropriate package must be downloaded and integrated into CSAR structure.
%Furthermore, all depended packages needed for new packages must also be added recursively.\\
This document considers the development of the solution to this problem through the predownload of the necessary data.
Different methods of encapsulation of CSARs will be defined.
It will be described the architecture of the software solution which will recognize external dependencies in a CSAR, eliminate them, resupply the CSAR with all the data necessary for deployment and also change the internal structure of the CSAR to display the achieved self-containment.
The prototype of the software will be implemented and validated.
%In addition some aspects of implementation will be described and explained.
\cleardoublepage


% BEGIN: Verzeichnisse

\iftex4ht
\else
\microtypesetup{protrusion=false}
\fi

%%%
% Literaturverzeichnis ins TOC mit aufnehmen, aber nur wenn nichts anderes mehr hilft!
% \addcontentsline{toc}{chapter}{Literaturverzeichnis}
%
% oder zB
%\addcontentsline{toc}{section}{Abkürzungsverzeichnis}
%
%%%

%Produce table of contents
%
%In case you have trouble with headings reaching into the page numbers, enable the following three lines.
%Hint by http://golatex.de/inhaltsverzeichnis-schreibt-ueber-rand-t3106.html
%
%\makeatletter
%\renewcommand{\@pnumwidth}{2em}
%\makeatother
%    
\tableofcontents

% Bei einem ungünstigen Seitenumbruch im Inhaltsverzeichnis, kann dieser mit
% \addtocontents{toc}{\protect\newpage}
% an der passenden Stelle im Fließtext erzwungen werden.

\listoffigures
%\listoftables

%Wird nur bei Verwendung von der lstlisting-Umgebung mit dem "caption"-Parameter benoetigt
%\lstlistoflistings 
%ansonsten:
\ifdeutsch
\listof{Listing}{Verzeichnis der Listings}
\else
\listof{Listing}{List of Listings}
\fi

%mittels \newfloat wurde die Algorithmus-Gleitumgebung definiert.
%Mit folgendem Befehl werden alle floats dieses Typs ausgegeben
\ifdeutsch
\listof{Algorithmus}{Verzeichnis der Algorithmen}
\else
%\listof{Algorithmus}{List of Algorithms}
\fi
%\listofalgorithms %Ist nur für Algorithmen, die mittels \begin{algorithm} umschlossen werden, nötig

% Abkürzungsverzeichnis
\printnoidxglossaries

\iftex4ht     
\else
%Optischen Randausgleich und Grauwertkorrektur wieder aktivieren
\microtypesetup{protrusion=true}
\fi

% END: Verzeichnisse


\renewcommand*{\chapterpagestyle}{scrplain}
\pagestyle{scrheadings}
\input{preambel/pagestyle}
%
%
% ** Hier wird der Text eingebunden **
%
% !TeX spellcheck = en_US

\chapter{Introduction}
\if 0
индустриальная революция?
\fi
Cloud applications market is increasing with great speed. 
Global annual growth is about 15\%~\cite*{statista_global}.
Furthermore one can observe the growth in the number of firms which are using Cloud applications. 
And it concerns not only some big corporations but also many small companies~\cite*{destatis_2014, destatis_2016}. \\ 
One of the most important reasons for the development of Cloud applications is the economy of resources.
It is much easier and often cheaper to rent a part of another's big mainframe than to maintain one's own server.
%It is also easier and cheaper to send a small package by mail than to keep your own car (server) and driver (administrator) for rare traffic as well.\\ 
The growing popularity of Cloud applications makes the automation and the ease of management increasingly important.
Management is understood as deployment, administration, maintenance and the final roll-off of Cloud applications.~\cite*{autocloud} \\   
The common problem of Cloud applications is a \emph{vendor lock-in}.~\cite*{lockin} 
The transfer of a Cloud application configured to interact with the \gls{api} of one provider to work with another provider and another \gls{api} is a difficult but important task. 
The ability to move a Cloud application to the more suitable provider quickly is a key to the development of competition and reducing the cost of maintenance.\\ %развитие конкуренции
\gls{tosca} \cite*{TOSCA-v1.0} is a standard to solve this problem. 
\gls{tosca} defines a meta-model to describing Cloud application's definition and management portable and interoperable. 
The use of TOSCA allows to simplify and automate the management of Cloud applications by different providers. 
According to \gls{tosca} standard a structure and management data are stored in a \textbf{C}loud \textbf{S}ervice \textbf{AR}chive (CSAR).
This archive contains the description of a Cloud application, its external functions, internal dependencies and the data for the deployment and operation.\\
OpenTOSCA \cite*{OpenTOSCA} is an open source constantly improving and expanding ecosystem for the TOSCA standard developed by the University of Stuttgart.
OpenTOSCA processes data in CSAR format and performs specified operations. %\\
Installation operations often contain links to external packages and programs which will be subsequently downloaded over the Internet for the deployment of a Cloud application.
These downloads can add expenses to the time required to download packages, money spent on rent of an idle server and Internet traffic for megabytes of pre-known data.
For many Cloud applications, this may mean a few seconds of delay. 
But for a large distributed application which contains a lot of identical nodes requiring the installation of the same external packages and programs, the costs can increase significantly.\\
The other problems of external dependencies are security and stability.
To ensure the security of information some firms restrict the access to Internet.
In other networks the Internet access is extremely limited.
For example, there can be no broadband access, slow communication only over a satellite at certain hours, etc.
An attempt to deploy a Cloud application with external dependencies in such networks may not succeed. \\
To solve these problems a software solution for removing external dependencies in CSARs will be developed and implemented during this work.
This software will analyze a CSAR, identify dependencies to external packages and resolve them by downloading the necessary data to install the package as well as data for all depended packages. 
Then all downloaded data will be added into the CSAR's structure to represent the changes made.
%The simplest example is to find in given CSAR all the commands like "apt-get install package", delete these commands, download the package and all depended packages and add them to the CSAR.
\\
This software must easily be expanded (in other words - to be a framework) since it is impossible to predict and describe all possible types of external dependencies.
The output of the framework is a CSAR which contains additions to the original structure, like all the packages necessary for the deployment of the Cloud application, with the minimum possible level of access to the Internet during operation.
%\clearpage 
\section*{Structure}
The work structure is as follows:
\begin{description}
\item[Chapter~\ref{chap:basis} -- \nameref{chap:basis}.] This chapter explains the basic terms of this work, which include definitions and descriptions of Cloud applications (section \ref{sec:cloud}), TOSCA standard (section \ref{sec:tosca}), OpenTOSCA environment  (section \ref{sec:opentosca}) and packet management (section \ref{sec:pm}).
\item[Chapter~\ref{chap:req} -- \nameref{chap:req}.] It clarifies requirements for the framework.
\item[Chapter~\ref{chap:conarch} -- \nameref{chap:conarch}.] The main concepts as well as architecture of the framework are explained and illustrated in chapter \ref{chap:conarch}.
\item[Chapter~\ref{chap:imp} -- \nameref{chap:imp}.] This chapter contains the description of the implementation.
 It explains the design and development of individual components of the software. 
\item[Chapter~\ref{chap:add} -- \nameref{chap:add}.] The new package manager will be added into the framework to proof the ease of extensibility. 
\item[Chapter~\ref{chap:check} -- \nameref{chap:check}.] In this chapter the output of the developed program will be presented and validated.
\item[Chapter~\ref{chap:zusfas} -- \nameref{chap:zusfas}.] The results of the work will be summarized in the last chapter.
\end{description}

%\input{...weitere Kapitel...}
% !TeX spellcheck = en_US

\chapter{Basis}
\label{chap:basis}
In this chapter, the fundamentals used in this work will be explained.
These include definitions for Cloud computing and Cloud applications, description of TOSCA standard and its implementation: OpenTOSCA.
At the end, a package management and tools for its automation are described.
\section{Cloud computing and Cloud application} \label{sec:cloud}
Understanding the problem requires a clear definition of the term "Cloud computing".
%In everyday life, you can often hear the phrase "Cloud computing", but what is it?\\
%\subsection{Definitions}
Unfortunately, a generally accepted definition of Cloud computing that describes all possible situations doesn't exist. 
But in the scientific community, the definition put forward by \gls{nist}~\cite*{wwwnist} is commonly used. 
This definition appropriately describes the concept of Cloud computing used in this paper, and therefore this definition will be used.\\
%\begin{definition}[Cloud computing]
\emph{\textbf{Cloud computing}\label{def:nist} is a model for enabling ubiquitous, convenient, on-demand network access to a shared pool of configurable computing resources (e.g., networks, servers, storage, applications, and services) that can be rapidly provisioned and released with minimal management effort or service provider interaction.}~\cite*{nist}\\
%\end{definition}
But computing is too abstract term, for our purpose we need something more practical, like an application.
Also, the are no generally accepted definitions of Cloud application, but it can be obtained from the definition of Cloud computing.\\
%\begin{definition}[Cloud application] 
\emph{A \textbf{Cloud application}\label{def:capp} is an application that is executed according to the Cloud computing model.} \\%~\cite*{cloudapp}\\
%\end{definition} 
In addition, a short meanings of a Cloud system, a provider, and a user will be provided.\\
%\begin{definition}[Cloud system] 
A composite Cloud application which consist of multiple small applications will be called a \textbf{Cloud system}\label{def:csys}. %\\
%\end{definition} 
%\subsubsection{Sides}
An owner of the physical platform, where Cloud computing takes place is called a \textbf{provider}.
An owner of the Cloud application renting a provider's platform is called a \textbf{user}.
\clearpage
\subsection*{Service Models}\label{def:servmod}
Cloud applications provide a wide range of different services.
Some groups of services which follow the common rules and perform the similar function are described by service models.
NIST distinguishes between three main types of such models.
\begin{itemize}
	\item  \gls{saas}. 
	The capability provided to the consumer is to use the provider’s applications running on a Cloud infrastructure. 
	The applications are accessible from various client devices through either a thin client interface, such as a web browser (e.g., web-based email), or a program interface. 
	The consumer does not manage or control the underlying Cloud infrastructure including network, servers, operating systems, storage, or even individual application capabilities, with the possible exception of limited userspecific application configuration settings.~\cite*{nist}
	\item \gls{paas}. 
	The capability provided to the consumer is to deploy onto the Cloud infrastructure consumer-created or acquired applications created using programming languages, libraries, services, and tools supported by the provider.
	The consumer does not manage or control the underlying Cloud infrastructure including network, servers, operating systems, or storage, but has control over the deployed applications and possibly configuration settings for the application-hosting environment.~\cite*{nist}
	\item \gls{iaas}.
	The capability provided to the consumer is to provision processing, storage, networks, and other fundamental computing resources where the
	consumer is able to deploy and run arbitrary software, which can include operating systems and applications.
	The consumer does not manage or control the underlying Cloud infrastructure but has control over operating systems, storage, and deployed applications; and possibly limited control of select networking components (e.g., host firewalls).~\cite*{nist}
\end{itemize}
%\subsection*{Deployment models}
%Similarly, NIST distinguishes between four types of deployment models.
%\begin{itemize}
%	\item Private cloud. 
%	The cloud infrastructure is provisioned for exclusive use by a single organization comprising multiple consumers (e.g., business units). It may be owned, managed, and operated by the organization, a third party, or some combination of them, and it may exist on or off premises.
%\item Community cloud.
%	The cloud infrastructure is provisioned for exclusive use by a specific community of consumers from organizations that have shared concerns (e.g., mission, security requirements, policy, and compliance considerations).
%	It may be owned, managed, and operated by one or more of the organizations in the community, a third party, or some combination of them, and it may exist on or off premises.
%\item Public cloud.
%	The cloud infrastructure is provisioned for open use by the general public. 
%	It may be owned, managed, and operated by a business, academic, or government organization, or some combination of them.
%	It exists on the premises of the cloud provider.
%\item Hybrid cloud. 
%	The cloud infrastructure is a composition of two or more distinct cloud infrastructures (private, community, or public) that remain unique entities, but are bound together by standardized or proprietary technology that enables data and application portability (e.g., cloud bursting for load balancing between clouds). 
%\end{itemize}
\subsection*{Usage of Cloud Computations}
Nowadays Cloud computing and applications can be found everywhere, and their number constantly grows~\cite*{cloud_stat}.
They are used for test and development, big data analyses, file storage and so on.
Cloud computing allows using resources effectively, to distribute the load to a system from several physical servers and to shift the maintenance to the providers. 
If  a service uses a single physical server and this server will be disabled, then the entire service will be completely unavailable too.
But if a Cloud application uses a hundred of physical servers, then disabling of one will not carry such serious consequences.
In addition, a user doesn't need to maintain a team of administrators for the event of various problems.\\
A user doesn't have a direct access to the infrastructure (servers and operating systems) when using a PaaS or a SaaS service models. He can operate only with the provided Application Programming Interface (API).
An API provides a set of methods to communicate with provider's infrastructure. 
Each provider defines his own set of methods, depending on his area of specialization. 
On the one hand, this specialization makes easier to work with the provider, but on the other hand, it becomes more difficult to redeploy an application to another provider.
\section{Topology and Orchestration Specification for Cloud	Applications} \label{sec:tosca}
%\subsection*{Definition}
The Topology and Orchestration Specification for Cloud Applications (\gls{tosca}) standard developed by \gls{oasis}~\cite{oasis} provides a new way to enable portable automated deployment and management of Cloud applications.
\gls{tosca} describes the structure of an application as a topology containing components and relationships between them.
%Plans capture management tasks by orchestrating management operations exposed by the components. \cite*{INBOOK-2014-01}
\gls{tosca} application is a Cloud application described according to the TOSCA standard.
This standard can be used not only to describe all stages of a Cloud application life-cycle but also to serve as a layer between the Cloud application and provider's API, allowing to implement a single application suitable for working with different providers. 
\subsection*{Structure of TOSCA Applications}
TOSCA specification provides a language to define components (described in section~\ref{def:servmod}) and relationships between them using $Service$ $Templates$. 
In addition it describes the management procedures which create or modify services using orchestration processes.
The description of elements of a TOSCA structure used in this work is provided. \\
A $Service$ $Template$ is the main component in a \gls{tosca} structure. 
It defines the structure (Topology Template) and process models (Plans) of the service. 
There can be many Service Templates within one TOSCA application.
The combination of topology and orchestration in a Service Template defines what is needed to be preserved across deployments in different environments to enable interoperable deployment of Cloud services and their management throughout the complete lifecycle (e.g. scaling, patching, monitoring, etc.).
This is useful when an application is ported to alternative Cloud environments.~\cite{TOSCA-v1.0_book} \\ %\\
$Plans$ provide capabilities to manage Cloud applications, especially their creation and termination.
These components combine management capabilities to create high-level management tasks which can then be executed for fully automated deployment, configuration and other operations of the application.
Plans can be started by a user or fully automatically and call management operations of the nodes in the topology. %\\
A $Topology$ $Template$ describes the topology of a Cloud application, defining nodes (Node Templates) and relations between them (Relationship Templates). %\\
A $Node$ $Template$ instantiates a Node Type as a component of a service. 
A $Node$ $Type$ defines the properties of such a component and the operations available to manipulate the component.
A $Relationship$ $Template$ instantiates a Relationship Type as a relationship between Node Templates in a Topology Template. 
The Relationship Template indicates that two nodes are connected and defines the direction of the connection.
A $Relationship$ $Type$ defines semantics and properties of the relationship.\label{subs:reltype} %\\
A Node Type and Relationship Type can by instantiated multiple times.
Those types are like abstract classes in high-level programming languages and Templates are objects of those classes.\\
A simple cloud application for weather calculating can be considered to provide an example. 
The calculation is performed by a python script which requires a python environment that is hosted on an Ubuntu virtual server.
$Node$ $Types$ must be defined for the python script, python environment and Ubuntu server.  
These $Node$ $Types$ will describe available operations for defined components.
It will a $compute$ operation for the python script, an $install$ operation for the python environments and $deploy$ and $shutdown$ operations for the server.
Additionally, one must define $Relationship$ $Types$ for $requires$ and $hosted$ $on$ dependencies.
Then these types will be instantiated inside the $Topology$ $Template$ named $weater$ $calculator$.
For each specified $Node$ $Type$, a corresponding $Node$ $Template$ with unique identifiers is created.
These identifiers are used by $Relationship$ $Templates$ to define the dependencies.
Figure~\ref{fig:weather} presents the described application. \\
Artifact represents the content necessary for a management such as executables (e.g. a script or an executable program), configuration files, data files, or something that might be needed for other executables (e.g. libraries or images of file system).
TOSCA distinguishes two kinds of artifacts: $Implementation$ $Artifacts$ and $Deployment$ $Artifacts$.
An $Implementation$ $Artifact$ represents the executable of an operation described by a Node Type.
An $Deployment$ $Artifact$ represents the executable for materializing instances of a node.
An $Artifact$ $Type$ describes a common type of an artifact: python script, installation package and so on.
An $Artifact$ $Template$ represents information about the artifact. 
For example the location of the artifact and other attendant data are stored in the $Artifact$ $Template$. 
As in the example with nodes above, types are like classes, templates are like objects, and artifacts represent content or a value of an object, but these values can't be changed. %\\
$Node$ $Type$ $Implementation$ defines the artifacts needed for implementing the corresponding Node Type.
If a Node Type contains $deploy$ and $shutdown$ operations, then the corresponding Node Type Implementation can contain two Implementation Artifacts with scripts for these operations and one Deployment Artifact with data necessary for the deployment. %\\
Implementations are like final classes between Node Types and Node Templates, but in TOSCA standard, the Implementation will be chosen only during execution.
Types, Templates, and Implementations defining a TOSCA application are stored in definition document which have the XML format. %\\
%%\subsection*{Usage}
The combination of topology and orchestration in a Service Template defines what is needed to be preserved across deployments in different environments to enable interoperable deployment of Cloud services and their management throughout the complete lifecycle (e.g. scaling, patching, monitoring, etc.).
This is useful when an application is ported to alternative Cloud environments.~\cite{TOSCA-v1.0_book} \\
% !TeX spellcheck = en_US

% We need layers to draw the block diagram
\usetikzlibrary{calc,positioning}
\usetikzlibrary{arrows.meta}

% Define a few styles and constants
\tikzstyle{entry}=[draw, fill=green!20, minimum height=2.5em, align=center]
\tikzstyle{text} = [above, text width=5em]
\tikzstyle{topology} = [entry, text width=18em, fill=white, 
minimum height=28em, rounded corners]
\tikzstyle{rtemplate} = [entry, text width=7em, shading = axis,rectangle, left color=blue!10!white, right color=blue!30!white,shading angle=135, anchor=north,
minimum height=10em, rounded corners]
\tikzstyle{item} = [entry, text width=5em, shading = axis,rectangle, left color=blue!10!white, right color=blue!30!white,shading angle=135, anchor=north,
minimum height=1.5em, rounded corners]
\tikzstyle{ntemplate} = [entry, text width=7em, shading = axis,rectangle, left color=blue!10!white, right color=blue!30!white,shading angle=135, anchor=north,
minimum height=8em, rounded corners]
\tikzstyle{rtype} = [entry, text width=6.5em, shading = axis,rectangle, left color=blue!10!white, right color=blue!30!white,shading angle=135, anchor=north,
minimum height=4em, rounded corners]
\tikzstyle{ntype} = [entry, text width=6.5em, shading = axis,rectangle, left color=blue!10!white, right color=blue!30!white,shading angle=135, anchor=north,
minimum height=7em, rounded corners]
\def\blockdist{2.3}
\def\edgedist{2.5}

\begin{figure}
	\centering
\begin{tikzpicture}
\node (topology) [topology] {};
\node at (topology.north)[yshift=-1em] {\large Topology Template};


\node (requirestemplate) [xshift=-4.5em, yshift=+9em] at ( topology) [rtemplate] {};
\node (requirestemplatetype) at (requirestemplate) [yshift=+1em] [item] {Type};
\node (requirestemplatefrom) at (requirestemplate) [yshift=-1em] [item] {From};
\node (requirestemplateto) at (requirestemplate) [yshift=-3em] [item] {To};
\node at (requirestemplate.north) [yshift=-2em][align=center] {Relationship\\Template\\\large Requires};

\node (hostedtemplate) [xshift=-4.5em, yshift=-2em] at ( topology) [rtemplate] {};
\node (hostedtemplatetype) at (hostedtemplate) [yshift=+1em] [item] {Type};
\node (hostedtemplatefrom) at (hostedtemplate) [yshift=-1em] [item] {From};
\node (hostedtemplateto) at (hostedtemplate) [yshift=-3em] [item] {To};
\node at (hostedtemplate.north) [yshift=-2em][align=center] {Relationship\\Template\\\large Hosted on};

\node (scripttemplate) [xshift=+4.5em, yshift=+11.5em] at ( topology) [ntemplate] {};
\node (scripttemplatetype) at (scripttemplate) [yshift=+0em] [item] {Type};
\node (scripttemplateID) at (scripttemplate) [yshift=-2em] [item] {ID};
\node at (scripttemplate.north) [yshift=-2em][align=center] {Node\\Template\\\large Scipt};

\node (environmenttemplate) [xshift=+4.5em, yshift=+3em] at ( topology) [ntemplate] {};
\node (environmenttemplatetype) at (environmenttemplate) [yshift=+0em] [item] {Type};
\node (environmenttemplateID) at (environmenttemplate) [yshift=-2em] [item] {ID};
\node at (environmenttemplate.north) [yshift=-2em][align=center] {Node\\Template\\\large Environment};

\node (servertemplate) [xshift=+4.5em, yshift=-5.5em] at ( topology) [ntemplate] {};
\node (servertemplatetype) at (servertemplate) [yshift=+0em] [item] {Type};
\node (servertemplateID) at (servertemplate) [yshift=-2em] [item] {ID};
\node at (servertemplate.north) [yshift=-2em][align=center] {Node\\Template\\\large Server};

\node (requirestype) [xshift=-5.5em, yshift=+2em] at ( requirestemplate.west) [rtype] {};
\node at (requirestype.north) [yshift=-2em][align=center] {Relationship\\Type\\\large Requires};

\node (hostedontype) [xshift=-5.5em, yshift=+2em] at ( hostedtemplate.west) [rtype] {};
\node at (hostedontype.north) [yshift=-2em][align=center] {Relationship\\Type\\\large Hosted on};

\node (scripttype) [xshift=+5.5em, yshift=+1.8em] at ( scripttemplate.east) [ntype, minimum height=5em] {};
\node  at (scripttype.south) [yshift=+2em] [item] {compute};
\node at (scripttype.north) [yshift=-1.5em][align=center] {Node Type\\\large Script};

\node (environmenttype) [xshift=+5.5em, yshift=+1.8em] at ( environmenttemplate.east) [ntype, minimum height=5em] {};
\node  at (environmenttype.south) [yshift=+2em] [item] {install};
\node at (environmenttype.north) [yshift=-1.5em][align=center] { Node Type\\\large Environment};

\node (servertype) [xshift=+5.5em, yshift=+3em] at ( servertemplate.east) [ntype] {};
\node  at (servertype.south) [yshift=+4em] [item] {deploy};
\node  at (servertype.south) [yshift=+2em] [item] {shutdown};
\node at (servertype.north) [yshift=-1.5em][align=center] { Node Type\\\large Server};

\draw [->,scale=5,line width=2pt,shorten <= -2pt] (requirestemplatefrom.north east) -- (scripttemplateID.south west);
\draw [->,scale=5,line width=2pt,shorten <= -2pt] (requirestemplateto.south east) -- (environmenttemplateID.north west);
\draw [->,scale=5,line width=2pt,shorten <= -2pt] (hostedtemplatefrom.north east) -- (environmenttemplateID.south west);
\draw [->,scale=5,line width=2pt] (hostedtemplateto.east) -- (servertemplateID.north west);

\draw [->,scale=5,line width=2pt] (requirestemplatetype.west) -- (requirestype.east);
\draw [->,scale=5,line width=2pt] (hostedtemplatetype.west) -- (hostedontype.east);

\draw [->,scale=5,line width=2pt] (servertemplatetype.east) -- (servertype.west);
\draw [->,scale=5,line width=2pt] (environmenttemplatetype.east) -- (environmenttype.west);
\draw [->,scale=5,line width=2pt] (scripttemplatetype.east) -- (scripttype.west);
\end{tikzpicture} 
\caption{Example: a cloud application for weather calculation} 	\label{fig:weather}
\end{figure}
%
\subsection*{CSAR} 
%
To store a TOSCA application a \gls{csar}\label{sec:csar} is used.
This is a ZIP-file with ".csar" extension that contains all the data needed for instantiation and management of TOSCA application.
They include definition documents, artifacts and so on.
In this form, a TOSCA application can be processed by a TOSCA runtime environment.\\
%\subsubsection*{Structure}
The root folder of any CSAR must contain the "Definitions" and "\gls{tosca}-Metadata" folders.
The "Definitions" folder contains definition documents one of which must define a Service Template.
The "\gls{tosca}-Metadata" folder must contain TOSCA metadata in the form of a file with the "TOSCA.meta" name.
This metafile consists of name/value pairs, one line for each pair. 
The first set of pairs describes CSAR itself (TOSCA version, CSAR version, creator and so on). 
All other pairs represent metadata of files from the CSAR. 
The metadata is used by a TOSCA runtime environment to process given files correctly.\\
%
%\subsubsection*{Terms}
%During this work, such terms as input CSAR and output CSAR will be used.
%The input \gls{csar} is the \gls{csar}, which can contain external references and will be processed by the framework. %\\
%The output CSAR is the CSAR, which was processed by the framework and doesn't contain external references. %(at least those that are handled by the framework).
\subsection*{Encapsulation of CSARs}
The encapsulation must be achieved through the download of external packages and generation of a new TOSCA node for each of them. 
But it can be interesting to analyze other techniques to encapsulate a CSAR. 
At first, we will described the methods not representing packages in a TOSCA topology and then the methods mirroring packages into the topology.\\
\subsubsection*{Generate Custom Repositories}
It's possible to download all necessary packages and create one's own custom package repository for each device used in the application. 
Then one must rework any package installation commands or exchange system preferences to setup an access to the custom repository.
This method introduces minimal changes in a TOSCA structure.
The main problem is the creation of the custom repositories. 
When a TOSCA application consists of many small devices with limited capabilities it can be difficult to start many big custom repositories.\\
\subsubsection*{Generate Shared Repository}
Another opportunity is to create a single repository for all devices in a TOSCA application.
It can be difficult to choose the right location for such a server, but since an application represents the connected system this step can redistribute the load to a more powerful device.
It is difficult to estimate the changes which will occur in a TOSCA topology while applying such a method.\\
\subsubsection*{One Node for One Package}
This method was suggested by IAAS. 
A new TOSCA node will be created for each downloaded package. 
All dependencies between packages will be mirrored to a TOSCA topology.
This is a very visual method, facilitating the understanding of a TOSCA application and dependencies between packages.\\
\subsubsection*{Sets of Packages}
A set of depended packages from a dependencies tree related to an external reference can be archived and represented in a TOSCA topology as a single node.
An installation of such a node will lead to the installation of all needed packages.
Of course, some packages can be saved in different archives redundantly, but a small size of a TOSCA application's structure will be achieved.
It will be impossible to trace dependencies (since all packages are represented by one node), but it can help to avoid a difficult structure which consists of hundreds of nodes. \\
\section{OpenTOSCA} \label{sec:opentosca}
OpenTOSCA provides an open source web-based ecosystem for \gls{tosca} applications. 
%It was developed at the University of Stuttgart in october 2012.
This ecosystem consists of three parts: the \gls{tosca} \textbf{runtime environment}, the graphical modeling TOSCA tool \textbf{Winery}, and the self-service portal for the applications available in the container \textbf{Vinothek}.~\cite*{OpenTOSCA}
Descriptions of the runtime environment and Winery will be provided in more detail. 
\subsection*{Runtime Environment}
The runtime environment enables a fully automated plan-based deployment and management of Cloud applications contained in a CSAR. 
The architecture of the environment is visualized by Figure~\ref{fig:openarch}.
Requests to the Container API are passed to the Control component, which orchestrates the different components, tracks their progress, and interprets the TOSCA application. 
The Core component offers common services to other components, e. g., managing data or validating XML.
Management operations of nodes and relationships are either provided by running (Web) services, e. g., the Amazon EC2 API, or by Implementation Artifacts contained in the CSAR.
In the latter case, the Implementation Artifact Engine is responsible to run these artifacts in order to make them available for plans. 
The plugin architecture of the Implementation Artifact Engine ensure extensibility.
Implementation Artifacts, e. g., a SOAP Web service implemented as Java Web archive, are processed by a corresponding plugin of the engine which knows where and how to run this kind of artifact. 
The plugins deploy the respective artifacts and return the endpoints of the deployed management operations to be stored in the Endpoints database.
The deployment of Web Archives on Tomcat~\cite*{tomcat} and Axis Archives on Apache Axis~\cite*{axis} is supported~\cite*{macharb}.
The Plan Engine handles plans in the same manner.
It is also build according to a plugin architecture and supports different workflow languages, e.g., Business Process Model and Notation (BPMN) or Business Process Execution (BPEL) Language, and their runtime environments.~\cite{INPROC-2013-45}
\\
A processing is done in following manner.  
First, the CSAR is unpacked and the files are put into the files store.
Then, the TOSCA definitions documents are loaded, resolved, validated, and processed by the Control component, which calls the Implementation Artifact Engine and the Plan Engine.
The Implementation Artifact Engine deploys the referenced Implementation Artifacts and stores their endpoints in the Endpoints database. 
Finally, the Plan Engine binds and deploys the application’s management plans.
The endpoints of the management plans are stored in the Plans database.~\cite{INPROC-2013-45}
%% !TeX spellcheck = en_US

% We need layers to draw the block diagram
\usetikzlibrary{calc,positioning}
\usetikzlibrary{arrows.meta}

% Define a few styles and constants
\tikzstyle{entry}=[draw, fill=green!20, minimum height=2.5em]
\tikzstyle{ann} = [above, text width=5em]
\tikzstyle{framework} = [entry, text width=30em, fill=white, 
minimum height=20em, rounded corners]
\tikzstyle{lang} = [entry, text width=9em, shading = axis,rectangle, left color=blue!10!white, right color=blue!30!white,shading angle=135, anchor=north,
minimum height=3em, rounded corners]
\tikzstyle{control} = [entry, text width=11em, shading = axis,rectangle, left color=blue!10!white, right color=blue!30!white,shading angle=135, anchor=north,
minimum height=12em, rounded corners]
\tikzstyle{core} = [entry, text width=14em, shading = axis,rectangle, left color=blue!10!white, right color=blue!30!white,shading angle=135, anchor=north,
minimum height=3em, rounded corners]
\tikzstyle{ia} = [entry, text width=9em, shading = axis,rectangle, left color=blue!10!white, right color=blue!30!white,shading angle=135, anchor=north,
minimum height=4em, rounded corners]
\tikzstyle{service} = [entry, text width=9em, shading = axis,rectangle, left color=blue!10!white, right color=blue!30!white,shading angle=135, anchor=north,
minimum height=4em, rounded corners]
\tikzstyle{plan} = [entry, text width=9em, shading = axis,rectangle, left color=blue!10!white, right color=blue!30!white,shading angle=135, anchor=north,
minimum height=4em, rounded corners]
\tikzstyle{item} = [entry, text width=8em, shading = axis,rectangle, left color=blue!20!white, right color=blue!40!white,shading angle=135, anchor=north,
minimum height=1em, rounded corners]
\def\blockdist{2.3}
\def\edgedist{2.5}

\begin{figure}
	\centering
\begin{tikzpicture}
\node (rr) [framework] {};


\node (lang1) [xshift=-30mm, yshift=+1.5em] at ( rr.north) [lang] {};
\node at (lang1) {\large Container API};

\node (control) [xshift=-30mm, yshift=+7em] at ( rr) [control] {};
\node [yshift=+1em] at (control.south) {\large Control};

\node [xshift=-0mm, yshift=+5em] at ( control) [item] {Process Files};
\node [xshift=-0mm, yshift=+3em] at ( control) [item] {Process TOSCA};
\node (impl)[xshift=-0mm, yshift=+1em] at ( control) [item] {Implementation Artifacts};
\node (plans)[xshift=-0mm, yshift=+-2em] at ( control) [item] {Plans};

\node (core) [xshift=-30mm, yshift=+3em] at ( rr.south) [core] {};
\node at (core) {\large Core};

\node (ia) [xshift=-10mm, yshift=+8em] at ( rr.east) [ia] {};
\node at (ia) {\large IA Engine};

\node (service) [xshift=-10mm, yshift=+2em] at ( rr.east) [service] {};
\node at (service) {\large Service Invoker};

\node (plan) [xshift=-10mm, yshift=-4em] at ( rr.east) [plan] {};
\node at (plan) {\large Plan Engine};

\draw [->,scale=5,line width=2pt] (lang1) -- (control);
\draw [->,scale=5,line width=2pt] (control) -- (core);
\draw [->,scale=5,line width=2pt] (plan.west) -- (core.east);
\draw [->,scale=5,line width=2pt] (service.west) -- (core.north east);
\draw [->,scale=5,line width=2pt] (ia.south west) -- (core.north east);
\draw [->,scale=5,line width=2pt] (impl.east) -- (ia.west);
\draw [->,scale=5,line width=2pt] (plans.east) -- (plan);
\end{tikzpicture} 
\caption{OpenTOSCA runtime environment control flow} 	\label{fig:opencontainer}
\end{figure}
% !TeX spellcheck = en_US

% We need layers to draw the block diagram
\usetikzlibrary{calc,positioning}
\usetikzlibrary{arrows.meta}

% Define a few styles and constants
\tikzstyle{entry}=[draw, minimum height=2em, align=center]
\tikzstyle{mytext}=[align=center]
\tikzstyle{frame} = [entry, text width=29em, fill=white,minimum height=20em, rounded corners]
\tikzstyle{db} = [entry, text width=26em, fill=white,minimum height=6em, rounded corners, left color=green!15!white, right color=green!20!white,shading angle=135, anchor=north]
\tikzstyle{dbentry} = [entry, text width=4em, fill=white,minimum height=3em, rounded corners, left color=blue!15!white, right color=blue!20!white,shading angle=135, anchor=north]
\tikzstyle{overbox} = [entry, text width=6em, shading = axis,rectangle, left color=blue!10!white, right color=blue!30!white,shading angle=135, anchor=north,
minimum height=3em, rounded corners]
\tikzstyle{longbox} = [entry, text width=24em, shading = axis,rectangle, left color=orange!15!white, right color=orange!20!white,shading angle=135, anchor=north,
minimum height=2em, rounded corners]
\tikzstyle{shortbox} = [entry, text width=12em, shading = axis,rectangle, left color=orange!15!white, right color=orange!20!white,shading angle=135, anchor=north,
minimum height=2em, rounded corners]
\tikzstyle{intern} = [entry, text width=8em, shading = axis,rectangle, left color=blue!10!white, right color=blue!30!white,shading angle=135, anchor=north,
minimum height=6em, rounded corners]
\tikzstyle{internr} = [entry, text width=17em, shading = axis,rectangle, left color=blue!10!white, right color=blue!30!white,shading angle=135, anchor=north,
minimum height=6em, rounded corners]
\tikzstyle{corned} = [entry, text width=5em, shading = axis,rectangle, left color=blue!10!white, right color=blue!30!white,shading angle=135, anchor=north,
minimum height=2em, rounded corners]
\def\blockdist{2.3}
\def\edgedist{2.5}

\begin{figure}
	\centering
\begin{tikzpicture}
\node (frame) [frame] {};

\node (conapi) [xshift=0em, yshift=+1em] at (frame.north) [longbox] {};
\node at (conapi.north) [yshift=-1em][mytext] {\large Container API};

\node (admininter) [xshift=0em, yshift=+4em] at (conapi.north) [overbox] {};
\node at (admininter.north) [yshift=-1.5em][mytext] {\large Admin UI};

\node (modeltool) [xshift=-8em, yshift=+4em] at (conapi.north) [overbox] {};
\node at (modeltool.north) [yshift=-1.5em][mytext] {\large Modelling\\\large Tool};

\node (selfport) [xshift=+8em, yshift=+4em] at (conapi.north) [overbox] {};
\node at (selfport.north) [yshift=-1.5em][mytext] {\large Self-Service\\\large Portal};

\node (control) [xshift=+5em, yshift=+8em] at (frame.west) [intern] {};
\node at (control.north) [yshift=-2em][mytext] {\large Control};

\node (core) [xshift=+0em, yshift=-1em] at (control.south) [intern] {};
\node at (core.north) [yshift=-2em][mytext] {\large Core};

\node (implart) [xshift=-8em, yshift=8em] at (frame.east) [internr] {};
\node at (implart.north) [xshift=-3em,yshift=-2em][mytext] {\large Implementation\\\large Artifact Engine};

\node (planengine) [xshift=+0em, yshift=-1em] at (implart.south) [internr] {};
\node at (planengine.north) [xshift=-3em,yshift=-2em][mytext] {\large Plan Engine};

\node (pluginimplart) [xshift=-2em, yshift=+2em] at (implart.east) [corned] {};
\node at (pluginimplart.north) [yshift=-1em][mytext] {\large plugin};

\node (dotsimplart) [xshift=-2em, yshift=-0.5em] at (implart.east) [corned] {};
\node at (dotsimplart.north) [yshift=-1em][mytext] {\large ...};

\node (pluginplanengine) [xshift=-2em, yshift=+2em] at (planengine.east) [corned] {};
\node at (pluginplanengine.north) [yshift=-1em][mytext] {\large plugin};

\node (dotsplanengine) [xshift=-2em, yshift=-0.5em] at (planengine.east) [corned] {};
\node at (dotsplanengine.north) [yshift=-1em][mytext] {\large ...};

\node (planport) [xshift=+3em, yshift=-1em] at (planengine.south) [shortbox] {};
\node at (planport.north) [yshift=-1em][mytext] {\large Plan Portability API};

\node (db) [xshift=-0em, yshift=+1em] at (frame.south) [db] {};
\node at (db.north) [yshift=-1em][mytext] {\large Database};

\node (plans) [xshift=-0em, yshift=+1em] at (db) [dbentry] {};
\node at (plans.north) [yshift=-1.5em][mytext] {Plans};

\node (files) [xshift=-5em, yshift=+1em] at (db) [dbentry] {};
\node at (files.north) [yshift=-1.5em][mytext] {Files};

\node (model) [xshift=-10em, yshift=+1em] at (db) [dbentry] {};
\node at (model.north) [yshift=-1.5em][mytext] {Model};

\node (endpoints) [xshift=+5em, yshift=+1em] at (db) [dbentry] {};
\node at (endpoints.north) [yshift=-1.5em][mytext] {Endpoints};

\node (Instancedata) [xshift=+10em, yshift=+1em] at (db) [dbentry] {};
\node at (Instancedata.north) [yshift=-1.5em][mytext] {Instance\\ data};



\end{tikzpicture} 
\caption{OpenTOSCA Architecture} 	\label{fig:openarch}
\end{figure}
\subsection*{Winery}\label{subs:wine}\label{tool:winery}
Winery provides a complete set of functions for graphically create, edit and delete elements in the TOSCA topology presented by a CSAR. 
It consists of four parts: the type and template management, the topology modeler, the BPMN4TOSCA plan modeler ~\cite{BPMN4TOSCA}, and the repository.\\
The type, template and artifact management enables managing all TOSCA types, templates and related artifacts. 
This includes node types, relationship types, policy types, artifact types, artifact templates, and artifacts such as virtual machine images.\\
The topology modeler allows to create service templates which consist of node templates and relationship templates. 
They can be annotated with requirements and capabilities, properties, and policies.\\
BPMN4TOSCA plan modeler offers web-based creation of BPMN models with the TOSCA extension: BPMN4TOSCA. 
That means the modeler supports the BPMN elements and structures required by TOSCA plans and not the full set of BPMN. 
The Stardust project~\cite*{stardust} offers Browser Modeler, which covers all phases of the Business Process Lifecycle including modeling, simulation, execution and monitoring. 
In the context of Winery, this modeler was extended to support BPMN4TOSCA.\\
The repository stores TOSCA models and allows managing their content. 
For instance, node types, policy types, and artifact templates are managed by the repository. 
The repository is also responsible for importing and exporting CSARs, the exchange format of TOSCA files and related artifacts.~\cite{winery} %\\
%Winery works under a Tomcat server and therefore a visual interface is available in a browser. %, example in Figure \ref{fig:winery_gui}.
An example of the TOSCA topology visualization is presented in Figure~\ref{fig:winery_source}.
%\begin{figure}[ht]   
%	\centering
%	\includegraphics[width=0.7\textwidth]{Screenshot_winery_gui.png}
%	\caption{Visual interface for $Winery$.}
%	\label{fig:winery_gui}
%\end{figure}
\begin{figure}[ht]   
\centering
\includegraphics[width=0.7\textwidth]{Screenshot_winery_source.png}
\caption{TOSCA topology visualized by $Winery$.}
\label{fig:winery_source}
\end{figure}
\section{Package Management} \label{sec:pm}
Packages and package management processes are described in this section.
An installation of packages from external source represents an external reference and therefore they will be considered in order to identify such references.
%\subsection*{Package managers}
Package is an archive file containing both data for installation of the program component and a set of metadata like name, function, version, producer, and a list of dependencies to other packages.~\cite*{opium}
These packages can present not only a complete program but also a certain component of a large application. % or libraries, a packages which can be used only by other programs.
For a user, a package manager is a set of software tools which automate the process of installing, updating, configuring and removing packages.
But from the operating system side, a package manager is used for managing the database of packages, their dependencies, and versions, to prevent erroneous installation of programs and missing dependencies.
%\subsection*{Packages}
%\subsection*{Dynamic libraries}
This task is especially complex in computer systems relying on dynamic library linking. 
Those systems share executable libraries of machine instructions across  applications. 
In these systems, complex relationships between different packages requiring different versions of libraries result in a challenge colloquially known as "dependency hell"~\cite*{linuxgeek}.
Good package management is vital to these systems.\\
%\subsection*{Repository}
To give users more control over the kinds of programs that they allow to install on their systems, packages are often downloaded only from a number of software repositories.
In Unix systems, a package manager uses official repositories appropriate for the operating system and the architecture  of device where it's operate, but it's possible to use additional repositories, like third-party repositories or repositories for another architecture.\\
%\subsection*{Dependencies} \label{subs:dep}
Package managers distinguish between two types of dependencies: $required$ and $preRequired$. %\\
Dependency $package1$ \textbf{$required$} $package2$ indicates that the $package2$ must be installed for a proper \textbf{operation} of the $package1$. %\\
Dependency $package1$ \textbf{$preRequired$} $package2$ indicates that the $package2$ must be installed for a proper \textbf{installation} of the $package1$. %\\
%An example for obtaining the dependency list for the Python package is shown in Listing~\ref{lst:dep}.
%\begin{lstlisting}[caption={Example of using $apt$-$cache$ to obtain dependency list for package python}\label{lst:dep},captionpos=t] 
%user@user:~$ apt-cache depends python
%python
%PreDepends: python-minimal
%Depends: python2.7
%Depends: libpython-stdlib
%Conflicts: <python-central>
%Breaks: update-manager-core
%Suggests: python-doc
%Suggests: python-tk
%Replaces: python-dev
%\end{lstlisting}
%\subsection*{Dependency tree}
In these examples, the $package2$ is needed for the $package1$, but the $package2$ itself can require additional packages.
A structure describing all necessary packages and dependencies between them for the given root-package is called a dependency tree. 
The dependency type $required$ can lead to cycles in dependency trees which differs them from the normal tree graph structures.
\subsection*{Example Dependencies Handling}
The $apt$-$get$ package manager will be considered to provide an example of a dependencies handling.
This application is part of \gls{apt} program which uses $dpkg$ application to communicate with an operating system.
The system keeps a database of packages and their condition.
These relations are presented in Figure~\ref{fig:packages}.\\
$apt$-$get$ has many functions: install, remove, update, autoremove, download and so on.
We will consider the install, remove and autoremove operations to present the common algorithms of processing.
When a package manager becomes a $package$ installation command, it builds a dependencies tree for the $package$ and checks the possibility to install these depended packages.
For example, it must check the compatibility with previously installed packages. 
If the check was successful, the $apt$-$get$ downloads and installs the packages starting with the bottom of the tree.
The $package$ is marked in the database as manually installed and all the other packages are marked as automatically installed. 
It will be helpful during the autoremove operation when all automatically installed packages will be checked whether they are still needed.
After installation of packages from the dependencies tree, the $package$ will be ready to work.\\
A $package$ can be deleted during the $remove$ $package$ command.
It happens only if there are no other packages depending on the $package$. 
If the deletion is very important, then these packages can also be removed too to keep the consistency of the database. 
The packages necessary for the $package$ itself will be deleted only by the autoremove command.
% !TeX spellcheck = en_US

% We need layers to draw the block diagram
\usetikzlibrary{calc,positioning}
\usetikzlibrary{arrows.meta}

% Define a few styles and constants
\tikzstyle{entry}=[draw, fill=green!20, minimum height=2.5em]
\tikzstyle{ann} = [above, text width=5em]
\tikzstyle{item} = [entry, text width=7em, shading = axis,rectangle, left color=blue!10!white, right color=blue!30!white,shading angle=135, anchor=north,
minimum height=2em, rounded corners, align=center]
%\def\blockdist{2.3}
%\def\edgedist{2.5}

\begin{figure}
	\centering
\begin{tikzpicture}
\node (aptget) [item] {apt-get};
\node (console) [yshift=+5em, xshift=+5em] at ( aptget) [item] {Console};
\node (script) [yshift=+5em, xshift=-5em] at ( aptget) [item] {Script};
\node (apt) [yshift=-3em] at ( aptget) [item] {APT};
\node (dpkg) [yshift=-3em] at (apt) [item] {dpkg};
\node (system) [yshift=-3em] at (dpkg) [item] {System database};

\draw [->,scale=1,line width=2pt] (console.south) -- (aptget);
\draw [->,scale=2,line width=2pt] (script.south) -- (aptget);
\draw [->,scale=2,line width=2pt] (aptget) -- (apt);
\draw [->,scale=2,line width=2pt] (apt) -- (dpkg);
\draw [->,scale=2,line width=2pt] (dpkg) -- (system);
\end{tikzpicture} 
\caption{Package management} 	\label{fig:packages}
\end{figure}
\section{Configuration Management Tools}
To ease the package management, various tools can be used. 
We will consider some of them to determine the form of files which contains external references.
Usually they are presented by commands in an executable file, which checks an environment and installs the necessary packages using a package manager.
Such files are called scripts and are commonly used.
Popular management tools like Bash, Ansible, Chef and CFEngine will be described below.
\subsection*{Bash} \label{lang:bash}
Bash is a Unix command language written as a free software.
It provides enough capabilities to be used as a  management tool.
In addition Bash denotes a command processor that typically runs in a text window, where a user types commands that cause actions.
Bash is examined because it is very popular since it is the default command line processor in Unix systems~\cite*{bashdef}.
Instead of typing commands direct into a command line, a script can be executed directly.~\cite{bash}
These scripts can be used to configure a system, install package, create files, check environment and so on.
Bash is a very popular and ease language, therefore a huge number of problems have solutions in Bash scripts already.
\subsection*{Ansible} \label{lang:ansible}
Ansible is an open-source automation engine that automates software provisioning, configuration management, and application deployment.
As with most configuration management software, Ansible has two types of servers: controlling machines and nodes.
First, there is a single controlling machine which is where orchestration begins.
Nodes are managed by a controlling machine over SSH.
The controlling machine describes the location of nodes through its inventory.
Ansible playbooks express configurations, deployment, and orchestration in Ansible.
The playbook format is YAML. 
Each playbook maps a group of hosts to a set of roles.
Each role is represented by calls to Ansible tasks.~\cite{ansible} 
\subsection*{Chef} \label{lang:shef}
Chef is a configuration management tool, which uses Ruby for writing system configuration files called "recipes".
They describe how Chef manages applications and utilities and how they are to be configured.
These recipes which can be grouped together as a "cookbook" for easier management define a series of resources that should be in a particular state: packages that should be installed, services that should be running, or files that should be written.
Chef can run in client/server mode, or in a standalone configuration named "chef-solo".
In client/server mode, the Chef client sends various attributes about the node to the Chef server. 
In solo mode the local system will be configured.~\cite{chef}
\subsection*{CFEngine}
CFEngine is an open source configuration management system.
Its primary function is to provide automated configuration and maintenance of large-scale computer systems, including the unified management of servers, desktops, consumer and industrial devices, embedded networked devices, mobile smartphones, and tablet computers.~\cite{cfengine}
Configurations are described by "policy" files, which are plain text-files with .cf extension.
These files define the necessary state of files, packages, users, processes, services and so on.~\cite{cfengine2}

% !TeX spellcheck = en_US

\chapter{Requirements}
\label{chap:req}
Since the main purpose of the developed framework is to Resolve References, further the $RR$ can be used as an abbreviation. 
$RR$ should eliminate external dependencies in a TOSCA topology represented by a CSAR file.
$RR$ must be easily extendable to provide the ability to eliminate a large number of dependency types.\\
As a first step, a minimal configuration which handles $Bash$ language with the $apt$-$get$ package manager and $Ansible$ language with the $apt$ package manager will be developed. 
These software handlers of languages and package managers will be called language modules and package manager modules.
As an example, the $Bash$ and $apt$-$get$ modules will remove package installation commands from bash-scripts ($apt$-$get$ $install$ \textbf{$package$}).
Then both the \textbf{$package$} itself and all the depended packages from his dependencies tree will be downloaded.
It is also necessary to update the topology of the TOSCA, by adding new nodes and dependencies.
To do so, common definitions will be added, like Relationship Types and Artifact Types.
 % for downloaded packages and dependencies from nodes previously containing an external reference to the node displaying the downloaded packages. 
Then new nodes will be defined by Node Types, Node Type Implementations, Artifacts Templates, and instantiated by Node Templates. 
Relations between nodes will be instantiated by Relationship Templates.
These Templates must be added to the right Service templates, where the nodes containing external references are instantiated.
%Dependencies between downloaded packages, representing dependency tree should be added too.
To find the Service Templates and Node Types corresponding to a certain artifact, it can be useful to apply preprocessing to the entire TOSCA topology. \\
%References chain can be build:\\
%$script$ $\rightarrow$ $Artifact$ $Implementation$ $\rightarrow$ $Node$ $Type$ $Implementation$ $\rightarrow$ $Node$ $Type$ $\rightarrow$ $Node$ $Template$ $\rightarrow$ $Service$ $Template$\\
After implementing the minimal configuration, it should be easy to add more language modules and package manager modules, like $Aptitude$ for Bash or completely new language like $Chef$.
In order to proof the correctness of the corresponding TOSCA topology, Winery described in section \ref{tool:winery} will be used.
\section*{Stages of the processing}
Here an example is provided, representing how the framework should work.
\begin{itemize}  
	\item Begin  \\
%	To start the work $RR$ needs input CSAR name, output CSAR name, and architecture of target hardware. 
%	This will be done using user input.
	An input CSAR will be extracted.
	\item Preprocessing\\
	During preprocessing stage, RR needs to analyze internal references.
	In additional, common Tosca definitions for artifacts and relations between packages will be added.
	\item Processing with language modules\\
	Each file from the input CSAR will be processed by Language modules.
	\item Processing with packet manager modules\\
    If the file belongs to an Language, it will be processed by the packet manager module belonging to the Language to find and resolve external references.
    Package name from this reference will be moved forward.
	\item Package handling\\
	Using the package name the package will be downloaded and TOSCA definitions created. These actions will be recursively repeated for all dependent packages, creating the dependency tree in the TOSCA topology.
	\item Topology handling\\
	Using information about internal references and dependencies the TOSCA Topology will be updated by creating new Node and Reference Templates. 
	\item End\\
	Meta-file should be updated and all data packed back to the CSAR.
\end{itemize}
These steps will be represented by the modules described in section \ref{sec:arch} and implemented in chapter \ref{chap:imp}.

\section*{Result}
As a result of the work, an output CSAR will be received. 
This CSAR must have the same functionality as the input CSAR, but all external references to additional packages must be resolved.
The output CSAR must be able to be deployed properly without downloading these packages over the Internet. 
In additional, the topology for the packages must be mirrored from the package manager's database to the TOSCA topology.
% !TeX spellcheck = en_US

\chapter{Concept and Architecture}\label{chap:conarch}
In this chapter, the concept and the architecture of the framework which can satisfy the requirements will be explained and substantiated.
Solutions to some additional problems will be presented. 
\section{Concept}
In this section, the main concept of this work are described.
The general structure of the framework is visualized in Figure~\ref{fig:gen}. 
In section~\ref{subs:analyse}, it will be explained how to determine the Node Templates which use the given artifact.
Then language modules and package manager modules functionality will be presented.
In section~\ref{subs:repres}, it will be expressed how to create a new node for a TOSCA topology. 
After that, a problem of the determination the architecture of the final platform will be explained and a solution presented.
In addition, it will be described, how the results can be validated.
% !TeX spellcheck = en_US

\begin{figure}
	\centering
\begin{tikzpicture}
\node[draw] (in) at (0,1.5) {Input CSAR};
\node[draw] (pp) at (0,+0.5) {Preprocessing};
\node[draw] (lm) at (0,-0.5) {Language modules};
\node[draw] (pmm) at (-3,-2) {Package manager modules};
\node[draw] (dtm) at (+3,-2) {Download tool modules};
\node[draw] (ph) at (-1.5,-3) {Package handler};
\node[draw] (th) at (0,-4) {Topology handler};
\node[draw] (out) at (0,-5) {Output CSAR};
\draw [->] (in) -- (pp);
\draw [->] (pp) -- (lm);
\draw [->] (lm) -- (pmm);
\draw [->] (lm) -- (dtm);
\draw [->] (pmm) -- (ph);
\draw [->] (dtm) -- (th);
\draw [->] (ph) -- (th);
\draw [->] (th) -- (out);
\end{tikzpicture} 
\caption{The general description of the software's work flow} 	\label{fig:gen}
\end{figure}


\subsection{Analysis of a TOSCA-Topology}\label{subs:analyse}
To update the \gls{tosca} topology properly, it is necessary to add references from the nodes where external references were to the newly created nodes which resolve the external references. 
According to TOSCA standard, references between Node Templates can only be created in the same Service Template.  
That means that each Node Template which uses artifacts with external references must be found.
Furthermore, Service Template where these Node Templates are instantiated must be determined to create there a Node Template for the new nodes and reference them to the Node Templates with external references.
The pointers to artifacts are contained by Artifact Templates which are used by Node Type Implementations.
By composing all the information the simple references chain can be built:\\
$Artifact$ $\rightarrow$ $Artifact$~$Template$ $\rightarrow$ $Node$~$Type$~$Implementation$ $\rightarrow$ $Node$~$Type$ $\rightarrow$ $Node$~$Template$ $\rightarrow$ $Service$~$Template$\\
Now consider the references in more detail. 
\begin{itemize}
	\item $Artifact$ $\rightarrow$ $Artifact$ $Template$\\
	An Artifact can be referenced by several Artifact Templates. (Despite the fact that this is a bad practice.)
	\item  $Artifact$ $Template$ $\rightarrow$ $Node$ $Type$ $Implementation$ \\
	The same way an Artifact Template can be used by several Node Type Implementations.
	\item $Node$ $Type$ $Implementation$ $\rightarrow$ $Node$ $Type$ \\
	A Node Type Implementation can describe an implementation of only one Node Type.
	\item  $Node$ $Type$ $\rightarrow$ $Node$ $Template$\\
	Each Node Type can have any number of Node Templates.
	\item  $Node$ $Template$ $\rightarrow$ $Service$ $Template$\\
	But each Node Template is instantiated only once.
\end{itemize}
This structure can be described as a tree with an Artifact as a root, and Service Templates as leaves (The example is on Figure \ref{fig:script_serv}) and will be called the internal dependencies tree.\\
%Of course it is possible to move in opposite direction, starting from a Node and moving toward scripts, but this method brings additional complexity. 
There is an additional problem in the reference between a Node Type and a Node Type Implementation.
A Node Type can have several implementations, but which one will be used is determined only during the deployment. 
The chosen solution to this problem is to use each Node Type Implementation in the hope, that they will not conflict.\\
%The method presented above can uniquely determine Node Templates and Service Template for a given script.
%Of course it is not guaranteed that found Node Type Implementation will be used during deployment, but we can't do anything with this. 
The following steps to build the internal dependencies tree can be executed during the preprocessing.
\begin{itemize}
	\item Find all Artifact Templates to build references from Artifacts to Artifact Templates.
	\item Find all Node Type Implementations. Since they contain references both to the Node Type and to the Artifact Templates, so the dependency from Artifact to Node Types can be built.
	\item Find all Service Templates and all contained Node Templates they contain. Each Node Template contains a reference to Node Type what is useful for building a dependency from Artifact to Node Template.
\end{itemize} 
In this way the required internal dependencies tree with references $Artifact$ $\rightarrow$ $Node$~$Template$ and $Artifact$ $\rightarrow$ $Service$~$Template$ can be built.
% !TeX spellcheck = en_US
\usetikzlibrary{calc,arrows.meta,positioning}
\tikzset{
    every node/.style={font=\sffamily\small},
    main node/.style={shape=rectangle, rounded corners,
    	draw, align=center,
    	top color=white, bottom color=blue!20}
}

\begin{figure}
	\centering
\begin{tikzpicture}[sibling distance=14em,->,>={Stealth[round,sep]},shorten >=1pt,auto,node distance=25mm]
    \node[main node] (1) {Artifact}
    child { node[main node](3) {Artifact Template 1} 
    	child { node[main node] (6) {Node Type Template 1} 
    			child { node[main node] (8) {Node Type 1} 
    				child { node[main node]  (11) {Node Template 1} 
    					child { node[main node] (13) {Service Template 1}}  
    				}
    				child { node[main node]  (12) {Node Template N} 
    					child { node[main node] (14) {Service Template K}}
    				}
    			}
    	}
    	child { node[main node] (7) {Node Type Implementation M} 
    		child { node[main node] (9) {Node Type L} 
    			child { node  (10) {\ldots} }
    		}
    	}
    }
    child { node[main node] (4) {Artifact Template T}  
    child { node (5) [below =of 4]{\ldots}}
	};

    \node at ($(3)!.5!(4)$) {\ldots};
    \node at ($(6)!.5!(7)$) {\ldots};
    \node at ($(11)!.5!(12)$) {\ldots};
    
\end{tikzpicture} 
\caption{An example tree describing how to find Service Templates and Node Templates for the given script} 	\label{fig:script_serv}
\end{figure}

\subsection{Search for Artifacts}
Since external references are stored in artifacts, we need to find all of them in order to identify the references.
The first simple solution is to analyze the structure of TOSCA application and identify all artifacts used.
But this method brings an possibility to miss some artifacts because some of them can be called from other artifacts.
This case is presented by Figure~\ref{fig:artart}.
In this example Implementation Artifact Engine calls the "Artifact 1" which calls the "Artifact 2".
The "Artifact 2" isn't called by Implementation Artifact Engine  directly and therefore will not be considered by the method described above.
The found solution is to analyze all files presented in a input CSAR and resolve external references in all of them.
We need to describe methods to identify files not described in the TOSCA topology for each supported configuration management tool.
% !TeX spellcheck = en_US

% We need layers to draw the block diagram
\usetikzlibrary{calc,positioning}
\usetikzlibrary{arrows.meta}

% Define a few styles and constants
\tikzstyle{entry}=[draw, minimum height=2em, align=center]
\tikzstyle{mytext}=[align=center]
\tikzstyle{frame} = [entry, text width=28em, fill=white,minimum height=10em, rounded corners]
\tikzstyle{engine} = [entry, text width=7em, fill=white,minimum height=4em, rounded corners, left color=green!15!white, right color=green!20!white,shading angle=135, anchor=north]
\tikzstyle{store} = [entry, text width=16em, fill=white,minimum height=6em, rounded corners, left color=blue!15!white, right color=blue!20!white,shading angle=135, anchor=north]
\tikzstyle{art} = [entry, text width=5em, fill=white,minimum height=3em, rounded corners, left color=red!15!white, right color=red!20!white,shading angle=135, anchor=north]
\def\blockdist{2.3}
\def\edgedist{2.5}

\begin{figure}
	\centering
\begin{tikzpicture}
\node (frame) [frame, label={[shift={(+0ex,-5ex)}]north:{\Large Runtime Environment}}] {};

\node (engine) [xshift=+5em, yshift=+1em] at (frame.west) [engine] {\large Implementation Artifact Engine};
\node (storage) [xshift=-9em, yshift=+2em] at (frame.east) [store,label={[yshift=-2em]north:{\Large File storage}} ] {};

\node (art1) [xshift=+4em, yshift=+1em] at (storage.west) [art] {\large Artifact 1};
\node (art2) [xshift=-4em, yshift=+1em] at (storage.east) [art] {\large Artifact 2};

\draw [->,scale=5,line width=2pt] (engine.east) --node [text width=2.5cm,midway,above ] {~~~~~~calls} (art1);
\draw [->,scale=5,line width=2pt] (art1.east) --node [text width=2.5cm,midway,above] {~~~~~~~~calls} (art2);

\end{tikzpicture} 
\caption{Bad artifacts call sequence} 	\label{fig:artart}
\end{figure}

\subsection{Modules and Extensibility}
It is impossible to identify all types of external references, even when only one language and one package manager are used (an example in listing~\ref{alg:unreadable}).
\begin{Listing} 
	\caption{Unreadable bash script}
	\label{alg:unreadable}
\begin{lstlisting}
#!/bin/bash
set  line = abcdefgijklmnoprst
# The "line" contains a part of the alphabet
set  word1 = ${line:0:1}${line:14:1}${line:17:1} 
# The 1th, 15th and 18th letters of the "line" variable are stored into the "word1".
# "word1" will contain the "apt" string 
set  word2 = ${line:6:1}${line:4:1}${line:17:1}
# The 7th, 5th and 18th letters of the "line" variable are stored into the "word2".
# "word2" will contain the "get" string 
$word1-$word2 install package
# This is the "apt-get install package" command,
#		 but to determine that a good interpreter is needed.
\end{lstlisting}
\end{Listing}
Since this work is aimed at creating the easily expanded and supplemented tool, then only basic usage of package managers will be considered initially.\\
The framework should handle different languages, each of which can support various package managers.
The best solution is to develop a modular system, where modules handle different languages and package managers.
The framework will contain a  language modules and each language module will contain package managers modules.
A language module should filter files not belonging to the language and the accepted files will be transmitted to the corresponding package manager modules.
This principle can be illustrated by Figure \ref{fig:lang_pm}.\\
A package manager module resolves an external reference and transmits the package name from this reference to a package handler described in section~\ref{subs:archph}.\\
Ease of adding of new modules to the framework will prove the correctness of the architecture.\\
At the beginning the most popular combination must be developed: the $bash$ language with the $apt-get$ package manager.
This simple and powerful tool allows to install, delete or update the set of packages in one line of code.
A line-by-line parser which analyses scripts and finds the installation commands can parse such commands and will be implemented.
After the modules for this combination will be implemented and validated, new language and package manager modules can be added.
% !TeX spellcheck = en_US

% We need layers to draw the block diagram
\usetikzlibrary{calc,positioning}

% Define a few styles and constants
\tikzstyle{entry}=[draw, fill=green!20, minimum height=2.0em, text width=8.0em]
\tikzstyle{ann} = [above, text width=5em]
\tikzstyle{framework} = [entry, text width=35em, fill=red!20, 
minimum height=19em, rounded corners]
\tikzstyle{lang} = [entry, text width=9em, fill=blue!20, 
minimum height=16em, rounded corners]
\def\blockdist{2.3}
\def\edgedist{2.5}

\begin{figure}
	\centering
\begin{tikzpicture}
\node (rr) [framework] {};
\node [xshift=+5mm, yshift=-2mm, below right] at (rr.north west) {\large References resolver framework };


\node (lang1) at ([xshift=-52mm,yshift=-4mm]rr) [lang] {};
\node [xshift=+3mm, yshift=-2mm, below right] at (lang1.north west) [text width=6em] {\large Language module 1 };
\node (lang2) at ([xshift=-7mm,yshift=-4mm]rr) [lang] {};
\node [xshift=+3mm, yshift=-2mm, below right] at (lang2.north west) [text width=6em] {\large Language module 2 };
\node (langn) at ([xshift=+52mm,yshift=-4mm]rr) [lang] {};
\node [xshift=+3mm, yshift=-2mm, below right] at (langn.north west) [text width=6em] {\large Language module N };  
\node at ($(lang2)!.5!(langn)$) {\ldots};

\node (l1pm1) at ([yshift=+15mm]lang1) [entry] {Package manager module};
\node (l1pm2) at ([yshift=0mm]lang1) [entry] {Package manager module};
\node (l1pmn) at ([yshift=-25mm]lang1) [entry] {Package manager module};
\node at ($(l1pm2)!.5!(l1pmn)$) {\vdots};

\node (l2pm1) at ([yshift=+15mm]lang2) [entry] {Package manager module};
\node (l2pm2) at ([yshift=0mm]lang2) [entry] {Download tool module};
\node (l2pm3) at ([yshift=-13mm]lang2) [entry] {Package manager module};
\node (l2pmn) at ([yshift=-25mm]lang2) [entry] {Package manager module};
%\node at ($(l2pm2)!.5!(l2pmn)$) {\vdots};

\node (lnpm1) at ([yshift=+15mm]langn) [entry] {Package manager module};
\node (lnpm2) at ([yshift=0mm]langn) [entry] {Package manager module};
\node (lnpm3) at ([yshift=-13mm]langn) [entry] {Download tool module};
\node (lnpmn) at ([yshift=-25mm]langn) [entry] {Download tool module};
%\node at ($(lnpm2)!.5!(lnpmn)$) {\vdots};


\end{tikzpicture} 
\caption{An example scheme representing several language modules containing package manager modules} 	\label{fig:lang_pm}
\end{figure}

\subsection{Representing Downloaded Packages in a TOSCA-Topology} \label{subs:repres}
A package node denotes the defined and instantiated element of the \gls{tosca} topology, the purpose of which is to install the package.
The addition of new package nodes to the TOSCA topology can be divided into several steps.
\begin{itemize}
	\item One must add definitions for common elements like Artifact Types or Relationship Types. 
		This can be done once.
	\item The package node common definition will be represented by a Node Type. 
		It must contain the $install$ operation, which represents the capability to install the node.
		%There will be described that this node must be installed.
	\item Artifacts (the downloaded data and the installation script) will be referenced by Artifact Templates.
	\item A Node Type Implementation will combine the artifacts to implement the $install$ operation.
	\item A Node Template will instantiate the package node in the corresponding Service Templates.
		To determine the corresponding Service Template the autor will use the preprocessing described in the section~\ref{subs:analyse}.
	\item A Reference Template will provide topology information allowing the observer (a user or a runtime environment) to determine which nodes the package must be installed for.
		References will be created from the Node Template which needs the package to the Node Template of the created package nodes.
\end{itemize}
After an execution of these steps, a definition of a package node will be finished and this node can be used.

\subsection{Determining Architecture of the Final Platform} \label{finplatf}
It can be difficult to choose the architecture of the device where packages will be installed.
Unfortunately, it is impossible to analyze the structure of any CSAR and give an unambiguous answer to the question which architecture which node will be deployed on.
There are many pitfalls here.\\
A single Service Template can use several physical devices with different architectures.
Many Node Types and Node Templates instantiated on different platforms can refer to the same Implementation Artifact.
This way one simple Implementation Artifact with a bash script containing "$apt$-$get$ $install$ $python$" command can be deployed on different devices within one Service Template (for example with the arm, amd64 and i386 architectures) and will result in the loading and installation of three different packages. 
For an end user, the ability to use such a simple command is a huge advantage, but for the framework, it can greatly complicate the analysis.
The following methods of architecture selection were designed.
\begin{itemize}
	\item $Deployment$ $environment$ $analysis$\\
	The script can analyze the system where it was started (for example using the "$uname$~$-a$" command) and depending on the result, it will install the package corresponding to the system's architecture.
	\item $Unified$ $architecture$\\
	The architecture will be defined by the user for the whole CSAR.
	\item $Artifact$ $specific$ $architecture$\\
	The architecture will be defined for each artifact separately.
\end{itemize}
%\subsubsection*{Analysis of methods}
The $deployment$ $environment$ $analysis$, which at first sight seems to be the most reliable solution, brings many additional problems.
Packages for different platforms can differ not only by architecture but also by the version and the list of dependencies.
As a consequence, chaos may occur while  mirroring these different packages with different versions to the \gls{tosca} topology.
The only found robust solution is to create a set of archives for each installed package. Using this method, data for one architecture are stored into one archive.
Such archives will contain the entire dependency tree for the given package.
But this approach contradicts one of the main ideas of this work: the dependencies trees should be mapped to the topology.\\
The $artifact$ $specific$ $architecture$ method carries an additional complexity to the user of the framework.
It will make a user to analyze each artifact and decide which architecture it will be executed on. 
This can be complicated by the fact that the same artifact can be executed on different architectures.\\
The method of the $unified$ $architecture$ was chosen as the simplest and easiest to implement.
If it will be necessary, this method can easily be expanded to the $artifact$ $specific$ $architectures$ method (by removing the user input at start, and choosing an architecture for each artifact separately) or to $deployment$ $environment$ $analysis$ (by downloading packages for all available architectures and adding the architecture determining algorithm to the installation scripts).

%\subsection{Extensibility}
%The framework should handle different languages, each of them can support various package managers.
%An language module should filter files not belonging to the language, accepted files will be processed 
%This principle can be illustrated by a Figure \ref{fig:lang_pm}.
%% !TeX spellcheck = en_US

% We need layers to draw the block diagram
\usetikzlibrary{calc,positioning}

% Define a few styles and constants
\tikzstyle{entry}=[draw, fill=green!20, minimum height=2.0em, text width=8.0em]
\tikzstyle{ann} = [above, text width=5em]
\tikzstyle{framework} = [entry, text width=35em, fill=red!20, 
minimum height=19em, rounded corners]
\tikzstyle{lang} = [entry, text width=9em, fill=blue!20, 
minimum height=16em, rounded corners]
\def\blockdist{2.3}
\def\edgedist{2.5}

\begin{figure}
	\centering
\begin{tikzpicture}
\node (rr) [framework] {};
\node [xshift=+5mm, yshift=-2mm, below right] at (rr.north west) {\large References resolver framework };


\node (lang1) at ([xshift=-52mm,yshift=-4mm]rr) [lang] {};
\node [xshift=+3mm, yshift=-2mm, below right] at (lang1.north west) [text width=6em] {\large Language module 1 };
\node (lang2) at ([xshift=-7mm,yshift=-4mm]rr) [lang] {};
\node [xshift=+3mm, yshift=-2mm, below right] at (lang2.north west) [text width=6em] {\large Language module 2 };
\node (langn) at ([xshift=+52mm,yshift=-4mm]rr) [lang] {};
\node [xshift=+3mm, yshift=-2mm, below right] at (langn.north west) [text width=6em] {\large Language module N };  
\node at ($(lang2)!.5!(langn)$) {\ldots};

\node (l1pm1) at ([yshift=+15mm]lang1) [entry] {Package manager module};
\node (l1pm2) at ([yshift=0mm]lang1) [entry] {Package manager module};
\node (l1pmn) at ([yshift=-25mm]lang1) [entry] {Package manager module};
\node at ($(l1pm2)!.5!(l1pmn)$) {\vdots};

\node (l2pm1) at ([yshift=+15mm]lang2) [entry] {Package manager module};
\node (l2pm2) at ([yshift=0mm]lang2) [entry] {Download tool module};
\node (l2pm3) at ([yshift=-13mm]lang2) [entry] {Package manager module};
\node (l2pmn) at ([yshift=-25mm]lang2) [entry] {Package manager module};
%\node at ($(l2pm2)!.5!(l2pmn)$) {\vdots};

\node (lnpm1) at ([yshift=+15mm]langn) [entry] {Package manager module};
\node (lnpm2) at ([yshift=0mm]langn) [entry] {Package manager module};
\node (lnpm3) at ([yshift=-13mm]langn) [entry] {Download tool module};
\node (lnpmn) at ([yshift=-25mm]langn) [entry] {Download tool module};
%\node at ($(lnpm2)!.5!(lnpmn)$) {\vdots};


\end{tikzpicture} 
\caption{An example scheme representing several language modules containing package manager modules} 	\label{fig:lang_pm}
\end{figure}

\subsection{Validation}
Checking the output of the framework is an important stage in the development of the program.
It is necessary to verify both the internal correctness of the output \gls{csar} and the possibility to deploy generated package nodes.
The validity of internal dependencies can be checked by $Winery$ tool from OpenTOSCA.
This tool for creating and editing CSAR archives is also great for visualizing the results.
Checking the deployment of the generated package nodes can be done manually by entering commands which start the  artifact's execution.

\subsection{Single Node Mode}
Besides the required mode of operation, the sets of packages mode described in section~\ref{mode:setsofpkg} will be  developed to provide possibility to generate relatively small encapsulated applications.
This additional mode will be called the single node mode.
To create the sets properly, the framework must combine the information about all needed packages and then create a single TOSCA node for all of them.


\section{Architecture}\label{sec:arch}
This section will present the architecture of the framework and the detailed description of its elements.
The main elements are a \boldmath $CSAR$ $handler$, a $references$ $resolver$, $language$ $modules$, $package$ $manager$ $modules$, a $package$ $handler$, and a $topology$ $handler$. \unboldmath

\subsection{CSAR handler} \label{subs:casr_h}
The CSAR handler provides an access to a \gls{csar} and maintains its consistency. 
It describes the processes of adding the new files (to handle the metadata), \mbox{archiving/unarchiving}, and choosing the final platform architecture. \\
The input CSAR is initially archived and must be decompressed in order to handle the content first.
When all external references will be resolved, the content will be archived to an output CSAR.
A new name-value pair must be added to the metadata for each new file integrated into the CSAR during processing. 
The name represents the full path to of the file.
The value contains the type of the file. 
This type will be used by a runtime environment to chose the right behavior. %\\ %\\
As already was mentioned in section~\ref{finplatf}, an architecture of the final platform will be chosen for the entirely CSAR.
A command line interface must be provided for a user, to allow him to chose the architecture. 
The chosen architecture must be saved to the CSAR for the case of future processing by the framework to avoid the collisions between architectures of packages.

\subsection{References Resolver} \label{subs:RR}
References Resolver is the main element, whose execution is divided into three stages: $preprocessing$, $processing$, $finishing$. \\
During the $preprocessing$ stage, the CSAR will be unarchived, common files added, and internal dependencies trees generated.
Figure \ref{fig:preproc} illustrates those steps.
During the $processing$ stage, all $language$ $modules$ will be activated, the operation is described in more detail in the next section. %\\
To finish the work all results will be packed into the output CSAR during the $finishing$ stage.
% !TeX spellcheck = en_US

\usetikzlibrary{calc,arrows.meta,positioning,arrows}
\tikzset{
    every node/.style={font=\sffamily\small},
    main node/.style={shape=rectangle, rounded corners,
    	draw, align=center,
    	top color=white, bottom color=blue!20}
}

\tikzstyle{entry}=[draw, fill=green!20, minimum height=2.2em, text width=7em]
\tikzstyle{myentry}=[draw, fill=Dandelion!20, minimum height=2.5em, text width=7em]
\tikzstyle{ann} = [above, text width=5em]
\tikzstyle{frame} = [entry, text width=10em, fill=red!20, 
minimum height=17em, rounded corners]
\tikzstyle{csar_content} = [entry, text width=9em, fill=blue!20, 
minimum height=14em, rounded corners]
\tikzstyle{out_content} = [entry, text width=10em, fill=blue!20, 
minimum height=17em, rounded corners]
\def\blockdist{2.3}
\def\edgedist{2.5}
\begin{figure}
	\centering
\begin{tikzpicture}[sibling distance=9em,->,>={Stealth[round,sep]},shorten >=1pt,auto,node distance=25mm]


\node (csar_frame) [frame] {};
\node [xshift=+5mm, yshift=-2mm, below right] at (csar_frame.north west) {\large Input CSAR};

\node (csar_content) at ( [yshift=-3mm]csar_frame) [csar_content] {};
\node [xshift=+5mm, yshift=-2mm, below right] at (csar_content.north west) {\large Content};

\node (l2pm1) at ([yshift=+15mm]csar_content) [entry] {Definitions\\};
\node (l2pm2) at ([yshift=0mm]csar_content) [entry] {TOSCA-Metadata\\};
\node (l2pmn) at ([yshift=-22mm]csar_content) [entry] {Artifacts\\};
\node at ($(l2pm2)!.5!(l2pmn)$) {\vdots};
    
\node [right of=csar_frame, xshift=30mm](dc_frame) [csar_content] {};
\node [xshift=+0mm, yshift=-2mm, below right] at (dc_frame.north west) {\large Content of the CSAR };
\draw [->] (csar_frame) -- (dc_frame);
\node (2pm1) at ([yshift=+15mm]dc_frame) [entry] {Definitions\\};
\node (2pm2) at ([yshift=0mm]dc_frame) [entry] {TOSCA-Metadata\\};
\node (2pmn) at ([yshift=-22mm]dc_frame) [entry] {Artifacts\\};
\node at ($(2pm2)!.5!(2pmn)$) {\vdots};

\node [right of=dc_frame, xshift=30mm](my_frame) [out_content] {};
\node [xshift=+4mm, yshift=-2mm, below right] at (my_frame.north west) {\large Content with new files};
\draw [->] (dc_frame) -- (my_frame);
\node (3pmd) at ([yshift=+20mm]my_frame) [myentry] {New definitions\\};
\node (3pm1) at ([yshift=+6mm]my_frame) [entry] {Definitions\\};
\node (3pm2) at ([yshift=-8mm]my_frame) [entry] {TOSCA-Metadata\\};
\node (3pmn) at ([yshift=-28mm]my_frame) [entry] {Artifacts\\};
\node at ($(3pm2)!.5!(3pmn)$) {\vdots};


\node[below=of dc_frame,main node, node distance=10mm, xshift=-4mm, yshift=+7mm] (11) {Script 2}
child { node[main node, yshift=-10mm]  (n21) {Node Template 1} 
		child { node[main node] (s31) {Service Template 1}}  
}
child { node[main node, yshift=-10mm]  (n22) {Node Template N} 
	child { node[main node] (s32) {Service Template N}}  
};
\draw [dashed,->] (dc_frame) -- (11);
\node[below left=of dc_frame,main node,xshift=+10mm] (12) {Script 1}
child { node  {\ldots} 
};
\draw [dashed,->] (dc_frame) -- (12);
\node[below right=of dc_frame,main node, xshift=-10mm] (13) {Script K}
child { node  {\ldots} 
};
\draw [dashed,->] (dc_frame) -- (13);
\node at ($(11)!.5!(13)$) {\ldots};
\node at ($(n21)!.5!(n22)$) {\ldots};

\end{tikzpicture} 
\caption{Preprocessing: decompression, addition of common definitions and generation of internal dependencies trees} 	\label{fig:preproc}
\end{figure}

\subsection{Language Modules} \label{subs:archlm}
Each $language$ $module$ describes a handling of one language and chooses files written in the language.
It also contains a list of supported package manager modules.
Each language module must provide the capability to generate a TOSCA node for the given package and this node must use the same language to install the package.
For example a Bash module must provide capability to define new package nodes which use bash to install the packages.
This means that a script and definitions for Artifact Templates, a Node Type, and a Node Type Implementation should be created by a language module.\\
As it is already mentioned above, during the $processing$ stage a language module analyzes all files one by one and checks their belonging to the language. 
Any files not belonging to the described language are filtered out.
The remaining files are transferred to the language module's $package$ $manager$ $modules$.
For example, a $bash$ module will pass only files with $".sh"$ extension which start with the $"\#!/bin/bash"$ line.
An $ansible$ module should have an additional functionality to unpack zip archives where ansible playbooks can be stored.
Since ansible playbooks don't contain a specific header or marker, the single sign of ansible files is the "$.yml$" extension. 
In the single node mode it will collect names of required packages from the corresponding package manager modules and create a single TOSCA node for all of them.

\subsection{Package Manager Modules} \label{subs:archpmm}
In normal mode a $package$ $managers$ $module$ finds external references, resolves them and transmits the package name to the $package$ $handler$ described in the next section.
After transmitting the names to the package handler, it must return the list of all required packages back to the language module to transfer them farther back to a language module.\\
Figure~\ref{fig:lang_ph} illustrates data flow between language modules, package manager modules and the package handler.
To resolve an external reference a package manager module will parse the given file. 
In the case of the apt-get module for bash, the module will read a file line-by-line searching for the commands starting with "apt-get install".
Such commands must be commented out and their arguments should be divided into separate package names which will be transferred to the package handler. 
% !TeX spellcheck = en_US
\usetikzlibrary{calc,arrows.meta,positioning}
\tikzset{
    every node/.style={font=\sffamily\small},
    main node/.style={shape=rectangle, rounded corners,
    	draw, align=center,
    	top color=white, bottom color=blue!20},
    data node/.style={shape=rectangle,
    draw, align=center,
    top color=white, bottom color=red!20}
}

\begin{figure}
	\centering
\begin{tikzpicture}[->,>={Stealth[round,sep]},shorten >=1pt,auto,level 1/.style={sibling distance=10em,node distance=25mm},
level 2/.style={sibling distance=8em,node distance=30mm},
level 3/.style={sibling distance=12em,node distance=30mm},
level 4/.style={sibling distance=6em,node distance=25mm},
level 5/.style={sibling distance=8em,node distance=25mm},
level 6/.style={sibling distance=8em,node distance=25mm}]
    \node[data node] (1)  {All files from original CSAR}
    child { node[main node](2) {Language 1} 
    	child { node[data node](21) {Files accepted by\\ Language 1} 
    		child { node {\ldots}}}}
	child { node[main node] (3) {Language 2}
		child { node[data node](31) {Files accepted by\\ Language 2} 
			child { node[main node, yshift=-13mm](32) {Package manager 1\\for Language 2} 
				child { node[data node](36) {Updated\\ files}}
				child { node[data node](37) {Package\\ names}}}
			child { node[main node, yshift=-13mm, xshift=-3mm](33) {Package manager 2\\for Language 2}
				child { node[data node](34) {Updated\\ files}}
				child { node[data node](35) {Package\\ names}}}
			child { node[main node, yshift=-13mm, xshift=+3mm](34) {Package manager K\\for Language 2}
				child { node[data node](38) {Updated\\ files}}
				child { node[data node](39) {Package\\ names}}}
			}
		}
    child { node[main node] (4) {Language N}
    	child { node[data node](41) {Files accepted by\\ Language N} 
    		child { node {\ldots}}}}
	;

	\node [main node,below, yshift=-20mm] at (35) (ph) {Package handler};
    \node at ($(3)!.5!(4)$) {\ldots};
    \node at ($(33)!.5!(34)$) {\ldots};
    \draw [->] (35) --(ph);
    \draw [->] (37) --(ph);
    \draw [->] (39) --(ph);
    
\end{tikzpicture} 
\caption{The data flow scheme between language modules, package manager modules and the package handler.} 	\label{fig:lang_ph}
\end{figure}

\subsection{Package Handler} \label{subs:archph}
The $package$ $handler$ communicates with an operating system's package manager. 
It can download installation data, determine the type of dependency between packages and provide a list with dependent packages for a given package.
To download the installation data this component will use given package name and the architecture specified by the CSAR handler.
Then it transfers the package name to the $topology$ $handler$ and repeats the actions for all depended packages recursively. 
In the process it stores all names of packages from the dependency tree of original package and sends them back to the calling package manager module.
To download the data the command "apt-get download \textbf{package}" can be used. 
The architecture can be specified by a ":$architecture$" suffix, for example, a "package:arm" mean the package for the $arm$ architecture.
The list of dependencies will be obtained using the "apt-cache depends \textbf{package}" command. 
The output of such command should be parsed in order to extract names of depended packages.
Type of dependency can be achieved in the same manner.
Of course, in case of a fault during a download of a package, a user interface should be provided to find a solution.
It can be: retry the download, ignore the package, rename the package or even break the framework's execution.

\subsection{Topology Handler} \label{subs:archtop}
%This element should handle the TOSCA topology and it has two main tasks: to analyze the TOSCA topology during the preprocessing stage to create internal dependencies trees and to use those trees to create TOSCA definitions for Node Templates and Relationship Templates in the right places for the packages provided by the package handler.
This element should handle the TOSCA topology and it has two main tasks: create internal dependencies trees and generate TOSCA definitions for packages provided by the package handler.
To build the trees the analysis of the TOSCA topology will be used during the preprocessing stage.
This procedure was described in section~\ref{subs:analyse}.
The needed TOSCA definitions for new package node include Node Templates and Relationship Templates which were described in section~\ref{subs:repres}.
To create the definitions in the right places the generated internal dependencies trees will be used.
The internal dependencies trees must be updated to represent changes after addition of new Node Templates and Relationship Templates.
%$Topology$ $handler$ adds a package to the topology. 
%This includes adding new files and updating existing files. 
%Necessary steps were described in section \ref{subs:repres}.
% !TeX spellcheck = en_US

\chapter{Implementation}\label{chap:imp}
This chapter provides an information about the implementation of the framework and its elements, whose behavior was described in chapter \ref{chap:conarch}.
Java language was chosen, because of its simplicity and strength. 
In this language, the elements are represented by classes.
Java uses additional kind of packages which describe third-party modules and make programming easier. 
The used Java packages will be mentioned here and the necessary license will be listed in the "NOTICE.txt" file in the source code's root folder.

\section{Global elements}
This section describes the elements used throughout the whole framework's execution.
A ZIP handler provides a functionality to operate ZIP archives, a CSAR handler keeps an interface to interact with a CSAR and a Utils helps to solve problems common to other elements.

\subsection*{Zip handler}
This is a small element with straight functionality. 
It serves to pack and unpack ZIP archives which are used by the CSAR standard.
It was decided to use the $java$.$utils$.$zip$ package for this task.
The functions of archiving and unarchiving are called $zipIt$ and $unZipIt$ respectively. 

\subsection*{CSAR handler}
This element provides an interface to access the CSAR content and stores information about files associated with it.
The most valuable data are the name of a temporary extraction folder, the list of files from the input CSAR, the meta-file entry, and the architecture of the target platform.
All this data is encapsulated into the CSAR handler.
The set of public functions allowing to operate with this element is available.
\begin{itemize}
	\item $unpack$ and $pack$ functions are used to extract the CSAR into the temporary folder and pack the folder to the output CSAR. 
	These functions use the $ZIP$~$handler$.
	\item $getFiles$ returns the list with files presented by the input CSAR.
	\item $getFolder$ returns the path to the folder where the CSAR was extracted.
	\item $getArchitecture$ returns the chosen architecture of the target platform.
	\item $addFileToMeta$ adds information about the new file to the meta-data.
\end{itemize}
Here is an example usage of the element.
When the CSAR handler extracts the input CSAR to the temporary extraction folder during the $unpack$'s call, it saves the folder's name. 
Then other elements can use the $getFolder$ function to get this name and access the data.

\subsection*{Utils}
This class provides the $createFile$, $getPathLength$, and $correctName$ methods used by many other elements.
The main purpose of these functions is to make the code cleaner. \\
Using the $createFile$ an element can create a file with the given content.
The $getPathLength$ function returns the deep of the given file's path and it is very useful for creating references between files.\\
OpenTOSCA uses some limitations to names of TOSCA nodes. 
Those names can't contain slashes, dots and so on.
To obtain an acceptable name from a given name the function $correctName$ can be used.

\section{References resolver}
This is the main module which starts by framework startup and is executed into three stages: preprocessing, processing and finishing.

\subsection*{Preprocessing}
At the preprocessing stage, the CSAR is unpacked, common \gls{tosca} definitions are generated and internal dependencies trees are built. 
%
%\subsubsection*{Unpacking}
As the first step, a user interface is provided to get the names of the input CSAR, output CSAR and the architecture of the final platform.
To unpack the CSAR the function $unpack$ from the CSAR handler is used.\\
%
%\subsubsection*{Generating TOSCA Definitions}
The $javax$.$xml$.$bind$ package was chosen for creating the common TOSCA definition. 
This Java package allows to generate a description - Java class describing an XML document. 
Those documents contain the following definitions.
\begin{itemize}
	\item $DependsOn$ and $PreDependsOn$ describe Relationship Types %(Described in the section \nameref{subs:reltype})
	  between packages.% (described in the section \nameref{subs:dep}). 
	\item $Package$ $Artifact$ defines a deployment Artifact Type for a package installation data.
	\item $Script$ $Artifact$ specify an implementation Artifact Type for a script installing a package.
	\item $Ansible$ $Playbook$ represent a deployment Artifact Type for a package installation via an ansible playbook.
\end{itemize}
An example description of the $Script$ $Artifact$ can be found in listing~\ref{lst:scripttype}.
Each description is represented by a separate Java class.\\
%
%\subsubsection*{Build internal dependencies trees}\label{subs:imp_findintref}
%Internal dependencies are mainly used by the \nameref{subs:archtop}.
%Therefore, these two modules were combined within the one Java class named $Topology$~$Handler$.
To build internal dependencies trees the topology handler described in section \ref{sec:imptophan} is used. 

\subsection*{Processing}
During this stage, all language modules listed in the framework are started.
For the references resolver element that is only two strings of code, but they start the main functionality of the framework.
The languages modules check all files presented in the input CSAR. 
The list of these files is stored in the CSAR handler, a pointer to which the modules became during their instantiation and translate to the corresponding package manager modules.
This system allows the modules to access the CSAR's content at any time.
%Since the language modules are stored in $language$ variable, this simple stage can be presented by the listing~\ref{lst:start_lang}.
%\begin{Listing}
%\caption{The processing stage}
%\label{lst:start_lang}
%\begin{lstlisting}
%for (Language l : languages)
%	l.proceed(cr);
%\end{lstlisting}
%\end{Listing}


\subsection*{Finishing}
When all external references will be resolved, the framework can enter its last stage.
At this stage, the changed data should be packed into an output \gls{csar}, whose name was entered during the preprocessing stage.
The function $pack$ from the CSAR handler is used. 
After this operation, we become a more encapsulated CSAR.
Its level of Internet access during a deployment will be significantly lower.

\section{Language modules} 
This section will describe the language modules. %implementation of %TODO
%For this purpose serve \nameref{subs:archlm} and \nameref{subs:archpmm}.
Since the framework is initially oriented to easy extensibility, an abstract model for the modules will be defined, so that new modules can be added by implementing this model.
The implementation of the bash and ansible modules will be provided at the end of the section.

\subsection*{Language model}
To specify the common functionality and behavior of different language modules, the language model is used. 
In Java, this model is described by an abstract class. 
The abstract class $Language$ is presented in listing~\ref{lst:langabst}.
The common variables for all language modules are the name of the language, the list with package manager modules, and the extensions of files.
And here are the common functions presented: 
\begin{itemize}
	\item $getName$ returns the name of this language.
	\item $getExtensions$ returns the list of extensions for this language.
	\item $proceed$ checks all original files.  
	Files written in the language should be transferred to every supported package manager module.
	\item $getNodeName$ uses a package name to generate the name for a Node Type, which will install the package using this language.
	\item $createTOSCA\_Node$ creates the definitions for a TOSCA node. 
	Since the created TOSCA nodes must install packages using the same language as the original node, all languages must provide the method for creating such definitions.
\end{itemize}
New language modules must be inherited from the language model and then can be added to the framework.

\subsection*{Bash module implementation}
The processing of the popular language was implemented. 
The bash module should accept only files written on the bash language.
To chose such files some signs inherent to all bash scripts can be used. 
These signs can be the file extensions (".sh" or ".bash") and the first line ("\#!/bin/bash"). 
Each file which contains those signs will be passed to supported package managers modules, in our case to the $apt$-$get$ module described later. \\
The bash module must provide a capability for the given package to create a definition of a TOSCA node which uses the bash language to install the package.
Such a bash TOSCA node is defined by the Package Type, the  Implementation, the  Artifact, and the Script Artifact.
The Package Type is a Node Type with an "$install$" operation and a name from the $getNodeName$ function.
The Package Implementation is a Node Type Implementation which refers the Package Artifact and the Script Artifact to implement the Package Type's "$install$" operation.
The Package Artifact and the Script Artifact are Artifact Templates referencing the installation data and a bash installation script respectively.
The installation script contains the bash header and an installation command, like "$dpkg$ -$i$ \textbf{installation\_data}".
The topology handler will instantiate the package node later by defining a Node Template.
Those definitions and the installation script are created by the $createTOSCA\_Node$ function.
%To avoid creating of the same nodes, the names of created nodes are stored in the $created\_packages$ list.
%Then the node name is generated using $getNodeName$ and TOSCA definitions for this name are created.

\subsection*{Ansible implementation}
To test the extensibility of the framework, the ansible language was added.
Since ansible playbooks are often packed into archives, it may be necessary to unpack them first and then to analyze the content.
Thus, the files are either immediately transferred to the package manager modules, or they are unzipped first.
Listing~\ref{lst:ansible_proceed} represents those operations.
As a sign of the ansible language, the ".$yml$" extension is used, since its playbooks don't contain any specific header.\\
Creating an ansible \gls{tosca} node for a package is a complicated operation. 
As the first step, the original files should be analyzed to determine the configuration (the set of options like a user name or a proxy server).
If the implemented analyzer is unable to find all necessary options, a user interface will be provided to fulfill any missing parameters.
Having the configuration a playbook and a configuration file will be created in a temporary folder.
After the installation data has been added to the folder, it can be packed to a zip archive.
This archive is an implementation artifact, which the Artifact Template should be created for.
A Node Type with an "$install$" operation % and a name built from the name of the package
should be defined.
And finally, a Node Type Implementation linking the operation and the Artifact Templates should be defined.
A Node Template will be added later by the topology handler.

\section{Package manager modules}
In this section, package manager modules will be specified.
Like languages, an abstract model will be defined to make the extensibility easier.
The apt-get module for bash and an apt module for ansible will be implemented.
\subsection*{Package manager model}
The model is described by an abstract class.
Its description contains only one function $proceed$ (in  listing~\ref{lst:pmabst}), that finds and eliminates external references, as well as passes the found package names to the package handler.
\subsection*{Apt-get for bash}
The apt-get package manager module is a simple line-by-line file parser which searches for the lines starting with the "$apt$-$get$ $install$" string, comments them out and passes this command's arguments to the package handler's public function $getPackage$. 
%The code can be found in the listing~\ref{lst:bash_apt_parse}.
\subsection*{Apt for ansible}
Since the package installation written in the $ansible$ language with the $apt$ package manager can be described in many different ways, then the processing will be a complicated task too.
It's worth mentioning that the processing uses a simple state machine and regular expression from the $java$.$util$.$regex$ package.

\section{Package Handler}
Package handler provides an interface for interaction with the package manager of the operating system.
It allows to download packages and to determine the type of dependencies between them.

\subsection*{Package downloading}
The download operation is performed using one recursive function $getPacket$. % defined in the listing \ref{lst:getpack}.
%\begin{Listing}
%\caption{The $getPackage$ definition}
%\label{lst:getpack}
%\begin{lstlisting}
%/**
%* Download package and check its dependency
%* 
%* @param language,  language name
%* @param packet, package name
%* @param listed, list with already included packages
%* @param source, name of package or file depending on the package
%* @param sourcefile, name of original file contained external reference.
%* @throws JAXBException
%* @throws IOException
%*/
%public void getPacket(Language language, String packet, List<String> listed, String source, String sourcefile)
%\end{lstlisting}
%\end{Listing}
The Arguments of the function will be defined shortly.
\begin{itemize}
	\item $language$ is a pointer to the language module which has accepted the original artifact.
	\item $packet$ is a name of the package.
	\item $listed$ holds a list with already downloaded packages.
	 It is not necessary to download them again, but new dependencies will be created.
	\item $source$ defines the parent element of the package. 
	It will be the original artifact file for the root package, and the depending package for other packages.
	\item $sourcefile$ is a name of the original artifact.
	%This name will be used by the $language$ to generate package node and by topology handler to create the dependency. 
\end{itemize}
This function downloads packages, calls the language's function $createTOSCA\_Node$ to create the TOSCA node for the package and the topology handler's functions $addDependencyToPacket$ or $addDependencyToArtifact$ to update the topology. Then this function calls itself recursively for all depended packages.
After those operations, a dependencies three for the $packet$ will be built.\\
The command $apt$-$get$ $download$ \emph{package} is used for downloading. 
If the process fails, a user input is provided to solve the problem. 
The user will be able to rename the package, ignore it or even break the processing.

\subsection*{Dependencies}
To determine the dependency type the $getDependensies$ function was developed.
It becomes a \emph{package} as an argument and uses the command $apt$-$cache$ $depends$ \emph{package} to build a list with dependencies for the \emph{package}. 
The $apt$-$cache$ command is a part of the $apt$-$get$ package manager and uses a packages database to print the dependencies.
The output is parsed to find strings like "Depends: \emph{dependent\_package}".
These dependent packages are combined to a list and returned back.
%An example output was presented in the section \ref{subs:dep}.

\section{Topology handling}\label{sec:imptophan}
The topology handler serves to update the TOSCA topology.
It builds the internal dependencies trees during the preprocessing stage.
Later the trees are used to find the right places for definitions of Node Templates and Dependency Templates.

\subsection*{Building internal dependencies trees}
At the preprocessing stage, this element analyzes all original definitions and constructs internal dependencies trees. %, as was described in the section~\ref{subs:analyse}.
To read those definitions from the XML files the package $org$.$w3c$.$dom$ was taken.\\
As the first step, all definitions of Artifact Templates are analyzed and pairs consist of an Artifact Template's ID and an artifact itself are built.
Then each Node Type Implementation will be read and Node Types and Artifact Template's IDs found. 
Now each artifact has a set with Node Types where it is used.
After the analysis of Service Templates, analog sets of Node Templates for each artifact will be created. 
In addition, for each Node Template one should keep a Service Templates, where this Node Template was defined.	

\subsection*{Updating Service Templates}
To update Service Templates two functions are provided.
\begin{itemize}
	\item $addDependencyToPacket(sourcePacket, targetPacket, dependencyType)$ generates a dependency between two package nodes.
	\item $addDependencyToArtifact(sourceArtifact, targetPacket)$ generates a dependency between the original node and a package node.
\end{itemize} 
Both functions find all Node Templates which instantiate the given $sourcePacket$ or $sourceArtifact$.
Besides, they find Service Templates where the Node Templates are defined.
The search is done with the help of the internal dependencies trees.
For each found Node Template a package node for the $targetPacket$ package should be instantiated by creating a new Node Template.
Then the dependency between the found Node Template and the new Node Template is created by defining a Relationship Template.
The Relationship Template references both Node Templates. 
The type of dependency is the value of the $dependencyType$ for the $addDependencyToPacket$ function and the $preDependsOn$ for the $addDependencyToArtifact$.\\
To update the existing TOSCA definition the $org$.$w3c$.$dom$ and $org$.$xml$.$sax$ packages are used. 
The definition of a new Node Template for the given $topology$ and $package$ is presented in listing \ref{lst:newnodetemp}.
\begin{Listing}
	\caption{Creating of a new Node Template}
	\label{lst:newnodetemp}
	\begin{lstlisting}  
	Element template = document.createElement("tosca_ns:NodeTemplate");
	template.setAttribute("xmlns:RRnt", RR_NodeType.Definitions.NodeType.targetNamespace);
	template.setAttribute("id", getID(package));
	template.setAttribute("name", package);
	template.setAttribute("type", "RRnt:" + RR_NodeType.getTypeName(package));
	topology.appendChild(template);
	\end{lstlisting}
\end{Listing}

% !TeX spellcheck = en_US

\chapter{Add new package manager module}\label{chap:add}
% !TeX spellcheck = en_US

\chapter{Validation}\label{chap:check}
In this chapter, the developed framework will be validated.
An input CSAR will be described in section~\ref{sec:inputcsar}.
The processing by the framework is described in section~\ref{sec:process}.
The output CSAR will be added to and displayed by Winery in section~\ref{sec:checkwin}.
Generated Artifacts will be checked in section~\ref{sec:checkart}.

\section{Input CSAR}\label{sec:inputcsar}
%In this test, a CSAR will be handled by the software. 
The handled CSAR provides a service for Automating the Provisioning of Analytics Tools based on Apache Flink.~\cite{csar_test}
The structure of the service is defined in Figure~\ref{fig:winery_source2}. 
The service uses a server virtualization environment named $vSphere$ (the $VSphere\_5.5$ node). 
In the environment works the $Ubuntu$ virtual server (the $Ubuntu$-$14.04$-$VM$ node).
The $Ubuntu$ hosts two applications: $Python$ (the $Python\_2.7$) and the $Flink$ $Simple$ (the $Flink\_Simple\_1.0.3$ node).
An analyze shows two external references. The $Python$ node installs the python package and the $Flink$ $Simple$ node - the Java package.
The service has two submodules: a Data Prediction and a Data Delivery, both are hosted on the $Flink$ $Simple$ node and require the Python node. 

\section{Processing}\label{sec:process}
%For Linux systems it can be easily installed.
To start the framework an Java environment is used.
After the start, a user should enter the input CSAR name, the output CSAR name, the architecture and chose the mode of operation.
After that, the framework works fully automatically, analyzing the artifacts and resolving any external references.
The output of the framework in the normal mode and the single node mode will be called a normal CSAR and a single node CSAR respectively.
Figure~\ref{fig:process} provides the example.
\begin{figure}[ht]   
	\centering
	\includegraphics[width=0.7\textwidth]{Screenshot_processing}
	\caption{Processing by the framework.}
	\label{fig:process}
\end{figure}

\section{Displaying with Winery}\label{sec:checkwin}
Winery was installed to test the correctness of the output CSAR. 
 This is an environment for the development of TOSCA systems and is useful for checking the results. %\\
 The input CSAR's representation by Winery is displayed in Figure~\ref{fig:winery_source2}.
 Those external references will be resolved by the framework and exchanged by new nodes in a output CSAR. 
 \begin{figure}[ht]   
 	\centering
 	\includegraphics[width=0.7\textwidth]{Screenshot_winery_source2}
 	\caption{Source CSAR represented by $Winery$.}
 	\label{fig:winery_source2}
 \end{figure}
   
 \subsection*{Add to Winery}
 The output CSAR is added to Winery.
 Due to a significant increase in size of the normal CSAR, this can be a fairly lengthy procedure.
 It was only six nodes in the input CSAR, but after the processing, the output CSAR contains more than 100 of nodes.
 Addition of a single node CSAR is faster process.
 The number of additional nodes coincides with the number of artifacts containing external references.
 During the addition to Winery, the CSAR's syntax is tested.
 In a case of errors, messages will be displayed.
 
 \subsection*{Display by Winery}
 The output CSARs where displayed by Winery.
Due to the high number of nodes, the processing of the normal CSAR can take a long time. 
 At the time, the  internal references are validated.
 If something was defined not properly, these erroneous nodes or links between them will not be displayed.
The representation of the normal CSAR is shown on Figure~\ref{fig:winery_output}, but only the part of the CSAR is visible.\\
 The structure seems very difficult to follow.
 To verify the topology some nodes was moved manually. 
 Figure~\ref{fig:winery_output2} displays the result. 
 And figure~\ref{fig:winery_output_single} visualize the single node CSAR with manually moved nodes. 
 The correctness of dependencies was verified by checking several nodes with the $apt$-$cache$ $depends$ command.
 By opening the content of the new nodes, it was verified, that there are right artifacts.
 For example it was noted that python installing node from the single node CSAR contains more then 50 Deployment Artifacts and one Implementation Artifact which installs corresponding packages.
 \begin{figure}[ht]   
 	\centering
 	\includegraphics[width=0.7\textwidth]{Screenshot_winery_output}  
 	\caption{The CSAR processed in normal mode and represented by Winery.}
 	\label{fig:winery_output}
 \end{figure}
 \begin{figure}[ht]   
 	\centering
 	\includegraphics[width=0.7\textwidth]{Screenshot_winery_output2}
 	\caption{The normal CSAR represented by Winery with some nodes moved manually.}
 	\label{fig:winery_output2}
 \end{figure}
\begin{figure}[ht]   
	\centering
	\includegraphics[width=0.7\textwidth]{Screenshot_winery_output_single}
	\caption{The representation of the CSAR processed in the single node mode.}
	\label{fig:winery_output_single}
\end{figure}
\section{Validate Artifacts}\label{sec:checkart}
It is necessary to check whether it is possible to install new packages using the generated artifacts.
At first Bash scripts will be tested, then ansible playbooks.

\subsection*{Validate Bash Scripts}
Since the Bash is used in the Linux's command line, it will be pretty easy to check Bash installation scripts by starting them.
Of course that must be done with the necessary privileges.
An example of the $python2.7$ installation is presented in listing~\ref{lst:check_bash_script}. %\\
The process ended without any warnings or errors, which means that it was completed successfully.
This way any Bash installation script can be checked.
\begin{Listing}
	\caption{Check Bash installation script}
	\label{lst:check_bash_script}
	\begin{lstlisting}
	user@user:~$ sudo RR_python2_7-minimal.sh 
	(Reading database ... 286091 files and directories currently installed.)
	Preparing to unpack python2_7-minimal.deb ...
	Unpacking python2.7-minimal (2.7.12-1ubuntu0~16.04.1) over (2.7.12-1ubuntu0~16.04.1) ...
	Setting up python2.7-minimal (2.7.12-1ubuntu0~16.04.1) ...
	Processing triggers for man-db (2.7.5-1) ...
	\end{lstlisting}
\end{Listing}

\subsection*{Validate Ansible Playbooks}
To check an ansible playbook we need to extract the zip file containing the playbook manually. 
During the regular execution, this work will be done by a runtime environment.
The call of the ansible runtime which proceeds the playbook is a simple procedure too.
An example is provided in Figure~\ref{fig:ansible_output2}. %\\
$Ok$ signals that the installation was completed successfully.
\begin{figure}[ht]   
	\centering
	\includegraphics[width=0.7\textwidth]{Screenshot_ansible_output}
	\caption{An ansible playbook's execution process}
	\label{fig:ansible_output2}
\end{figure}
% !TeX spellcheck = en_US

\chapter{Summary}\label{chap:zusfas}


%
%\input{content/latex-tipps}
%% !TeX spellcheck = en_US
\allowdisplaybreaks
\chapter*{Listings}\label{chap:listing}
%TODO correct listings
\begin{Listing} 
	\caption{Generate the Artifact Type definition for scripts}
	\label{lst:scripttype}
\begin{lstlisting}
public class RR_ScriptArtifactType {

@XmlRootElement(name = "tosca:Definitions")
@XmlAccessorType(XmlAccessType.PUBLIC_MEMBER)
public static class Definitions {

@XmlElement(name = "tosca:ArtifactType", required = true)
public ArtifactType artifactType;

@XmlAttribute(name = "xmlns:tosca", required = true)
public static final String tosca="http://docs.oasis-open.org/tosca/ns/2011/12";
@XmlAttribute(name = "xmlns:winery", required = true)
public static final String winery="http://www.opentosca.org/winery/extensions/tosca/2013/02/12";
@XmlAttribute(name = "xmlns:ns0", required = true)
public static final String ns0="http://www.eclipse.org/winery/model/selfservice";
@XmlAttribute(name = "id", required = true)
public static final String id="winery-defs-for_tbt-RR_ScriptArtifact";
@XmlAttribute(name = "targetNamespace", required = true)
public static final String targetNamespace="http://docs.oasis-open.org/tosca/ns/2011/12/ToscaBaseTypes"; 

public Definitions() {
artifactType = new ArtifactType();
}

public static class ArtifactType {
@XmlAttribute(name = "name", required = true)
public static final String name = "RR_ScriptArtifact";
@XmlAttribute(name = "targetNamespace", required = true)
public static final String targetNamespace="http://docs.oasis-open.org/tosca/ns/2011/12/ToscaBaseTypes"; 
ArtifactType() {}
}
}


}
\end{lstlisting}
\end{Listing}

\begin{Listing} 
	\caption{Generate definition for Script Artifact Type}
	\label{lst:scripttype_create}
	\begin{lstlisting}
// output filename
public static final String filename = "RR_ScriptArtifact.tosca";

/**
* Create ScriptType xml description
* 
* @param cr
* @throws JAXBException
* @throws IOException
*/
public static void init(Control_references cr) throws JAXBException,
IOException {
	File dir = new File(cr.getFolder() + Control_references.Definitions);
	dir.mkdirs();
	File temp = new File(cr.getFolder() + Control_references.Definitions + filename);
	if (temp.exists())
	temp.delete();
	temp.createNewFile();
	OutputStream output = new FileOutputStream(cr.getFolder()
	+ Control_references.Definitions + filename);
	
	JAXBContext jc = JAXBContext.newInstance(Definitions.class);
	
	Definitions shema = new Definitions();
	
	Marshaller marshaller = jc.createMarshaller();
	marshaller.setProperty(Marshaller.JAXB_FORMATTED_OUTPUT, true);
	marshaller.marshal(shema, output);
	cr.metaFile.addFileToMeta(Control_references.Definitions + filename, "application/vnd.oasis.tosca.definitions");
}
\end{lstlisting}
\end{Listing}
\begin{Listing}
\caption{Abstract language model}
\label{lst:langabst}
\begin{lstlisting}
public abstract class Language {
	
	// List of package managers supported by language
	protected List<PacketManager> packetManagers;
	
	// Extensions for this language
	protected List<String> extensions;
	
	// Language Name
	protected String Name;
	
	// To access package topology
	protected Control_references cr;
	
	// List with already created packages
	protected List <String> created_packages;

	/**	Generate node name for specific packages
	* @param packet
	* @param source
	* @return
	*/
	public abstract String getNodeName(String packet, String source);
	
	
	/**	Generate Node for TOSCA Topology
	* @param packet
	* @param source
	* @return
	* @throws IOException
	* @throws JAXBException
	*/
	public abstract String createTOSCA_Node(String packet, String source) throws IOException, JAXBException;
}
\end{lstlisting}
\end{Listing}

\begin{Listing}
\caption{Abstract package manager model}
\label{lst:pmabst}
\begin{lstlisting}
public abstract class PacketManager {

// Name of manager
static public String Name;

protected Language language;

protected Control_references cr;

/**
* Proceed given file with different source (like archive)
* 
* @param filename
* @param cr
* @param source
* @throws FileNotFoundException
* @throws IOException
* @throws JAXBException
*/
public abstract void proceed(String filename, String source) throws FileNotFoundException, IOException,
JAXBException;
}
\end{lstlisting}
\end{Listing}

\begin{Listing}
\caption{Create TOSCA node for bash language}
\label{lst:create_bash}
\begin{lstlisting}
	public String createTOSCA_Node(String packet, String source) throws IOException, JAXBException{
if(created_packages.contains(packet+"+"+source))
return packet;
created_packages.add(packet+"+"+source);
packet = getNodeName(packet, source);
RR_NodeType.createNodeType(cr, packet);
RR_ScriptArtifactTemplate.createScriptArtifact(cr, packet);
RR_PackageArtifactTemplate.createPackageArtifact(cr, packet);
RR_TypeImplementation.createNT_Impl(cr, packet);
return packet;
}
\end{lstlisting}
\end{Listing}

%\begin{Listing}
%\begin{lstlisting}[caption={Create TOSCA node for bash language}\label{lst:create_bash},captionpos=t] 
%\end{lstlisting}
%\end{Listing}

\begin{Listing}
\caption{File parsing for Bash + apt-get }
\label{lst:bash_apt_parse}
\begin{lstlisting}
public void proceed(String filename, String source)
throws IOException, JAXBException {
String prefix = "";
for (int i = 0; i < Utils.getPathLength(filename) - 1; i++)
prefix = prefix + "../";
if (cr == null)
throw new NullPointerException();
System.out.println(Name + " proceed " + filename);
BufferedReader br = new BufferedReader(new FileReader(filename));
boolean isChanged = false;
String line = null;
String newFile = "";
while ((line = br.readLine()) != null) {
// split string to words
String[] words = line.replaceAll("[;&]", "").split("\\s+");
// skip space at the beginning of string
int i = 0;
if (words[i].equals(""))
i = 1;
// look for apt-get
if (words.length >= 1 + i && words[i].equals("apt-get")) {
// apt-get found
if (words.length >= 3 + i && words[1 + i].equals("install")) {
// replace "apt-get install" by "dpkg -i"
System.out.println("apt-get found:" + line);
isChanged = true;
for (int packet = 2 + i; packet < words.length; packet++) {
System.out.println("packet: " + words[packet]);
//							cr.AddDependenciesScript(source, words[packet]);
cr.getPacket(language, words[packet], source);
}
}
newFile += "#//References resolver//" + line + '\n';
} else
newFile += line + '\n';
}
br.close();
if (isChanged) {
// references found, need to replace file
// delete old
File file = new File(filename);
file.delete();

// create new file
FileWriter wScript = new FileWriter(file);
wScript.write(newFile, 0, newFile.length());
wScript.close();
}
}
\end{lstlisting}
\end{Listing}

\begin{Listing}
\caption{Ansible proceeding}
\label{lst:ansible_proceed}
\begin{lstlisting}
	public void proceed(Control_references cr) throws FileNotFoundException,
IOException, JAXBException {
	if (cr == null)
	throw new NullPointerException();
	for (String f : cr.getFiles())
	for (String suf : extensions)
	if (f.toLowerCase().endsWith(suf.toLowerCase())) {
		if (suf.equals(".zip")) {
			proceedZIP(f);
		} else
		proceed(f, f);
	}
}

/**
* proceed given file
* 
* @param filename
* @param cr
* @param source
*            of file, example - archive
* @throws FileNotFoundException
* @throws IOException
* @throws JAXBException
*/
public void proceed(String filename, String source)
throws FileNotFoundException, IOException, JAXBException {
	for (PacketManager pm : packetManagers)
	pm.proceed(filename, source);
}

/**
* Handle ZIP package
* 
* @param zipfile
* @throws FileNotFoundException
* @throws IOException
* @throws JAXBException
*/
private void proceedZIP(String zipfile) throws FileNotFoundException,
IOException, JAXBException {
	boolean isChanged = false;
	// String filename = new File(f).getName();
	String folder = new File(cr.getFolder() + zipfile).getParent()
	+ File.separator + "temp_RR_ansible_folder" + File.separator;
	List<String> files = zip.unZipIt(cr.getFolder() + zipfile, folder);
	for (String file : files)
	if (file.toLowerCase().endsWith("yml"))
	proceed(folder + file, zipfile);
	if (isChanged) {
		new File(cr.getFolder() + zipfile).delete();
		zip.zipIt(cr.getFolder() + zipfile, folder);
	}
	zip.delete(new File(folder));
	
}
\end{lstlisting}
\end{Listing}
%
%\renewcommand{\appendixtocname}{Anhang}
%\renewcommand{\appendixname}{Anhang}
%\renewcommand{\appendixpagename}{Anhang}
\appendix

\clearpage

%\printindex

\printbibliography

\ifdeutsch
Alle URLs wurden zuletzt am 17.\,03.\,2008 geprüft.
\else
All links were last followed on July 30, 2017.
\fi

\pagestyle{empty}
\renewcommand*{\chapterpagestyle}{empty}
\Versicherung
\end{document}
